\documentclass{article}
\usepackage{amsmath}
\usepackage{amssymb}
\usepackage{amsthm}
\usepackage{amsfonts}
\usepackage{fancyhdr}


\pagestyle{fancy}
\fancyhead{}
\fancyhead[L]{}
\fancyhead[C]{Second-Order Recurrence Relations}
\fancyhead[R]{\today}

\theoremstyle{definition}
\newtheorem{definition}{Definition}

\theoremstyle{remark}
\newtheorem*{remark}{Remark}

\theoremstyle{plain}
\newtheorem{theorem}{Theorem}
\newtheorem{lemma}{Lemma}
\renewcommand\qedsymbol{$\blacksquare$}

\begin{document}
\begin{definition}
  Given a second-order linear homogenous recurrence relation with constant coefficients:
  \[
    a_k = Aa_{k-1} + Ba_{k-1} \hspace{0.1in} \forall k \in \mathbb{Z}\left(k \geq 2\right)
  \] 
  the \textbf{characteristic equation of the relation} is 
  \[
    t^2 - At - B = 0
  \]
\end{definition}

\begin{lemma}
  Let $A$ and $B$ be real numbers. A recurrence relation of the form 
  \[ a_k = Aa_{k-1} + Ba_{k-2} \hspace{0.1in} \forall k \in \mathbb{Z}\left(k\geq 2\right)\] 
  is satisfied by the sequence 
  \[ 1, t, t^2, t^3, \ldots, t^n, \ldots \] 
  where $t$ is a nonzero real number, if, and only if, $t$ satisfies the equation 
  \[ t^2 - At - B = 0 \]
\end{lemma}
\begin{lemma}
  If $r_0, r_1, r_2, \ldots$ and $s_0, s_1, s_2, \ldots$ are sequences that satify the same second-order linear homogenous recurrence relation with constant coefficients, and if $C$ and $D$ are \emph{any} numbers, then the sequence $a_0, a_1, a_2, \ldots$ defined by the formula 
  \[
    a_n = Cr_n + Ds_n \hspace{0.1in} \forall n \in \mathbb{Z}^{nonneg}
  \]
  also satisfies the same recurrence relation
\end{lemma}

\begin{lemma}
  Let $A$ and $B$ be real numbers and suppose that the characteristic equation 
  \[
   t^2 - At - B = 0 
  \]
  has a single root $r$. Then the sequences $1, r^1, r^2, r^3, \ldots, r^n, \ldots$ and $0, r, 2r^2, 3r^3, \ldots, nr^n, \ldots$ both satisfy the recurrence relation 
  \[
    a_k = Aa_{k-1} + Ba_{k-2} \hspace{0.1in} \forall n \in \mathbb{Z}\left(n \geq 2\right)
  \]
\end{lemma}

\begin{theorem}[Single-Root Theorem]
  Suppose a sequence $a_0, a_1, a_2, \ldots$ satisfies a recurrence relation 
  \[a_k = Aa_{k-1} + Ba_{k-2}\] 
  for some real numbers $A$ and $B$ with $B \neq 0$ and $\forall k \in \mathbb{Z}\left(k \geq 2\right)$. If the characteristic equation $t^2 - At - B = 0$ has a single (real) root $r$, then $a_0, a_1, a_2, \ldots$ is given by the explicit formula 
  \[a_n = Cr^{n} + Dnr^{n}\]
  where $C$ and $D$ are the real numbers whose values are determined by the values of $a_0$ and any other known value of the sequence
\end{theorem}

\begin{proof}
  Suppose for some real numbers $A$ and $B$, a sequence $a_0, a_1, a_2, \ldots$ satisfies the recurrence relation $a_k = Aa_{k-1} + Ba_{k-2} \hspace{0.1in} \forall k \in \mathbb{Z}\left(k\geq 2\right)$, and suppose the characteristic equation $t^2-At-B=0$ has one distinct root $r$. We will show that 
  \[
    \forall n \in \mathbb{Z}^{nonneg}, \hspace{0.1in} a_n = Cr^n + Dnr^n
  \]
  where $C$ and $D$ are numbers such that \\
  \[
    a_0 = Cr^{0} + D(0)r^{0} = C \hspace{0.2in} \text{and} \hspace{0.2in} a_1 = Cr^{1} + D(1)r^{1} = Cr + Dr
  \]
  Let $P(n)$ be the equation 
  \[
    a_n = Cr^{n} + Dnr^{n} 
  \]
  We use strong mathematical induction to prove that $P(n)$ is true for all integers $n \geq 0$. In the basis step, we prove that $P(0)$ and $P(1)$ are true. We do this because in the inductive step we need the equation to hold for $n=0$ and $n=1$ in order to prove that it holds for $n=1$

  \textbf{Show that $P(0)$ and $P(1)$ are true:} The truth of $P(0)$ and $P(1)$ is automatic because $C$ and $D$ are exactly those numbers that make the follwing equations true:
  \[
    a_0 = Cr^{0} + D(0)r^{0} = C \hspace{0.2in} \text{and} \hspace{0.2in} a_1 = Cr^{1} + D(1)r^{1} = Cr + Dr
  \]

  \textbf{Show that for all integers $k\geq 1$, if $P(i)$ is true for all integers $i$ from $0$ through $k$, then $P(k+1)$ is also true:} Suppose that $k\geq 1$ and for all integers $i$ from $0$ through $k$, 
  \[ a_i = Cr^{i} + Dir^{i}\] 
  We must show that 
  \[ a_{k+1} = Cr^{k+1} + D\left(k+1\right)r^{k+1}\]
  Now by the inductive hypothesis, 
  \[ a_k = Cr^{k} + Dkr^{k} \hspace{0.2in} \text{and} \hspace{0.2in} a_{k-1} = Cr^{k-1} + D\left(k-1\right)r^{k-1} \]
  so 
  \begin{align*}
    a_{k+1} &= Aa_{k} + Ba_{k-1} \tag{by definition of $a_0, a_1, a_2, \ldots$} \\ 
    &= A\left(Cr^{k} + Dkr^{k}\right) + B\left(Cr^{k-1} + D\left(k-1\right)r^{k-1}\right) \tag{by inductive hypothesis}\\ 
    &= C\left(Ar^{k} + Bkr^{k}\right) + D\left(Akr^{k} + B\left(k-1\right)r^{k-1}\right) \tag{by combining like terms} \\ 
    &= Cr^{k+1} + D\left(k+1\right)r^{k+1} \tag{by Lemma 1}
  \end{align*}
  This was what was to be shown. 

\end{proof}
\begin{remark}
  The reason the last equality follows from Lemma 2 is that since $r$ satisfies the characteristic equation and is the only root of it, the sequences $1, r^{1}, r^{2}, r^{3}, \ldots, r^{n}, \ldots$ and $0, r, 2r^{2}, 3r^{3}, \ldots, nr^{n}, \ldots$ satify the recurrence relation
\end{remark}
\end{document}
