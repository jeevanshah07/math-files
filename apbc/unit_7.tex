\documentclass[addpoints]{exam}
\usepackage{amsmath}
\usepackage{amsfonts}
\usepackage{tcolorbox}
\usepackage{tikz}
\usepackage{hyperref}
\usepackage{pgfplots}
\usepackage{mdframed}


%=============================================
        %POINTS FORMATTING
%=============================================
\marksnotpoints 
%if you would prefer the exam to say "points" instead of "marks", you can delete the above line.

%\pointformat{\textbf{(\thepoints)}} 
%delete comment if you want bold points

\pointsinrightmargin
%\marginpointname{ \points}
\bracketedpoints
%delete comment if you want point values to be printed in the right margin instead of at the start of the question. The same command with the "right" places the point values on the left margin.

%\
%=============================================
        %HEADER/FOOTER FORMATTING
%=============================================
\pagestyle{headandfoot}
\firstpageheadrule
\runningheadrule
\firstpageheader{Unit 7: Sequences and Series}{Convergence Tests}{Mr. Shah \\ BC Calc}
\runningheader{Unit 7}{Convergence Tests}{BC Calc}
\firstpagefooter{}{}{}
\runningfooter{ }{\thepage}{ }

\begin{document}

\vspace{1in}

\begin{center}
    \fbox{\parbox{6in}{\centering
    \vspace{0.25em}
    Summary of Convergence and Divergence Tests
    \rule{0.95\linewidth}{0.4pt}
    \vspace{0.5em}
    \begin{enumerate}
        \item Divergence Test:
        Given \(\displaystyle\, \sum\limits_{n=0}^\infty a_{n}\) does \(\displaystyle\, \lim \limits_{n\to\infty} a_{n} = 0\)? If it does not, then the series must diverge. \newline \textit{Note that the Divergence Test does not state anything about convergence, it simply tells whether a series diverges or not.}
        \item Direct Comparison Test: Given \(\displaystyle\, \sum\limits_{n=0}^\infty a_{n} \text{ and } \sum\limits_{n=0}^\infty b_{n}\) with \(0 \leq a_{n} \leq b_{n}\):
        \begin{enumerate}
            \item If \(\displaystyle \sum\limits_{n=0}^\infty a_{n}\) is divergent then so is \(\displaystyle\, \sum\limits_{n=0}^\infty b_{n}\)
            \item If \(\displaystyle \sum\limits_{n=0}^\infty b_{n}\) is convergent then so is \(\displaystyle\, \sum\limits_{n=0}^\infty a_{n}\)
        \end{enumerate}
        \item Limit comparison test: Given \(\displaystyle\, \sum\limits_{n=0}^\infty a_{n} \text{ and } \sum\limits_{n=0}^\infty b_{n}\) with \(a_{n}, b_{n} \geq 0\), let \(\displaystyle L = \lim \limits_{n\to\infty}\frac{a_n}{b_n}\). If \(L\) is finite and positive (ie \(0 < L < \infty\)) then \(\displaystyle \sum\limits_{n=0}^\infty a_n \text{ and } \sum\limits_{n=0}^\infty b_n\) have the same convergence \\
        \textit{Note as well that the limit can also be \(\displaystyle \lim\limits_{n\to\infty} \frac{b_n}{a_n}\)}
        \item Integral Test: Given \(\displaystyle\, \sum\limits_{n=0}^\infty a_{n}\) and:
        \begin{enumerate}
            \item There is a function \(f(x) \text{ such that } f(n) = a_n\)
            \item \(f(x)\) is continuous
            \item \(f(x)\) is positive 
            \item \(f(x)\) is decreasing 
        \end{enumerate}
        Then the convergence of \(\displaystyle\, \sum\limits_{n=0}^\infty a_n\) will be the same as the convergence of \(\displaystyle \int_{0}^\infty f(x)\, dx\) \\
        \textit{Note that the conditions of positive and decreasing must only be true \textbf{eventually}}
        \item Alternating Series Test: Given \(\displaystyle \sum\limits_{n=0}^\infty (-1)^{n}a_n \text{ or } \sum\limits_{n=0}^\infty (-1)^{n+1} a_{n}\) then if:
        \begin{enumerate}
            \item \(a_n\) is decreasing (ie. \(a_n \geq a_{n+1)}\))
            \item \(\displaystyle \lim\limits_{n\to\infty} a_n = 0\)
        \end{enumerate}
        \(\displaystyle \sum\limits_{n=0}^\infty (-1)^{n} a_n \text{ or } \sum\limits_{n=0}^\infty (-1)^{n+1} a_n\) converges
    \end{enumerate}
    }} 
\end{center}

\begin{center}
    \fbox{\parbox{6in}{\centering
    \vspace{0.25em}
    Summary of Convergence and Divergence Tests (cont.)
    \rule{0.95\linewidth}{0.4pt}
    \vspace{0.5em}
    \begin{enumerate}
        \setcounter{enumi}{5}
        \item Ratio Test: Given \(\displaystyle \sum\limits_{n=0}^\infty a_n\), let \(\displaystyle L =  \lim\limits_{n\to\infty} \left|\frac{a_{n+1}}{a_n}\right|\)
        \begin{enumerate}
            \item \(L < 1\): \(\displaystyle \sum\limits_{n=0}^\infty a_n\) converges
            \item \(L > 1\): \(\displaystyle \sum\limits_{n=0}^\infty a_n\) diverges
            \item \(L = 1\): Ratio test is inconclusive (series could converge or diverge)
        \end{enumerate} 
        \textit{Note that the ratio test will \textbf{always} be inconclusive for polynomials, so do not waste your time trying it if \(a_n\) is a polynomial}
        \item Root Test: Given \(\displaystyle \sum\limits_{n=0}^\infty a_n\), let \(\displaystyle L = \lim\limits_{n\to\infty} \sqrt[n]{\left| {a_n} \right|} = \lim\limits_{n\to\infty} \left|a_n\right|^{\frac{1}{n}}\)
        \begin{enumerate}
            \item \(L < 1\): \(\displaystyle \sum\limits_{n=0}^\infty a_n\) converges
            \item \(L > 1\): \(\displaystyle \sum\limits_{n=0}^\infty a_n\) diverges
            \item \(L = 1\): Root test is inconclusive (series could converge or diverge)
        \end{enumerate} 
        \textit{Note that the root test will \textbf{always} be inconclusive for polynomials, so do not waste your time trying it if \(a_n\) is a polynomial}
            \end{enumerate}
    }} 
\end{center}

\begin{center}
    \fbox{\parbox{6in}{\centering
    \vspace{0.25em}
    Alternating Series Remainder Theorem
    \rule{0.95\linewidth}{0.4pt} \vspace{0.5em} 
    Given \(\displaystyle \sum\limits_{n=0}^\infty (-1)^{n} a_n \text{ or } \sum\limits_{n=0}^\infty (-1)^{n+1} a_n\), and an approximation for the sum, \(S\), at \(n=N\), \(S_N\), the error between the actual sum and the approximation can by found by using:
    \newline \(\left|S-S_N\right| = \left|R_N\right| \leq a_{N+1}\)
    \newline meaning that the error between the true sum and the approximation must be less than or equal to the next neglected term (the next term not included in the approximation)
    }} 
\end{center}

\begin{center}
    \fbox{\parbox{6in}{\centering
    \vspace{0.25em}
    Special Series
    \rule{0.95\linewidth}{0.4pt}
    \vspace{0.5em}
    \begin{enumerate}
        \item P-series: Given \(\displaystyle \sum\limits_{n=0}^\infty \frac{1}{n^p}\)
        \begin{enumerate}
            \item If \(p > 1\) \(\displaystyle \sum\limits_{n=0}^\infty \frac{1}{n^p}\) will converge
            \item If \(p \leq 1\) \(\displaystyle \sum\limits_{n=0}^\infty \frac{1}{n^p}\) will diverge \\
        \end{enumerate}
        \textit{Note that if \(p=1\) then the series is known as the harmonic series}
        \item Geometric Seres: Given \(\displaystyle \sum\limits_{n=0}^\infty ar^{n} \text{ or } \sum\limits_{n=1}^\infty ar^{n-1}\) then,
        \begin{enumerate}
            \item If \(\left|r\right| < 1\) then \(\displaystyle \sum\limits_{n=0}^\infty ar^{n} \text{ or } \sum\limits_{n=1}^\infty ar^{n-1}\) converges and its sum can be found with \\
            \(\displaystyle
            S = \frac{a}{1-r}, \text{ where } a \text{ is the first term of the series}
            \)
            \item If \(\left|r\right| \geq 1\) then \(\displaystyle \sum\limits_{n=0}^\infty ar^{n} \text{ or } \sum\limits_{n=1}^\infty ar^{n-1}\) diverges and, therefore, its sum \textit{cannot} be found
        \end{enumerate}
        \item Telescoping Series: Given a series, \(\displaystyle \sum\limits_{n=0}^\infty a_n\), that has partial sums in the form \((s_1 - s_2) + (s_2 - s_3) + (s_3 - s_4) + ... + (s_n - s_{n+1})\) its sum can be found with,
        \(\displaystyle
        \lim\limits_{n\to\infty} s_1 - s_{n+1}
        \)
        however, this is only true if, and only if, \(s_n \to\) a finite number as \(n\to\infty\)
    \end{enumerate}
    }} 
\end{center}

\begin{tcolorbox}[title=ABSOLUTE/CONDITIONAL CONVERGENCE, colback=white, colframe=green!75!black]
\begin{itemize}
    \item \underline{Definition:} Let $\displaystyle\,\sum_{n=0}^{\infty} a_n$ be a \textbf{convergent series}. If $\displaystyle\,\sum_{n=0}^{\infty} |a_n|$ is \textit{convergent}, then $\displaystyle\,\sum_{n=0}^{\infty}$ is called \textbf{absolutely convergent}. If $\displaystyle\,\sum_{n=0}^{\infty} |a_n|$ is \textit{divergent}, then $\displaystyle\,\sum_{n=0}^{\infty} a_n$ is called \textbf{conditionally convergent}
    \item \underline{Tip:} Due to the fact that if $\displaystyle\sum_{n=0}^{\infty} |a_n|$ is convergent then so is $\displaystyle\,\sum_{n=0}^{\infty} a_n$, \textbf{always} test $\displaystyle\,\sum_{n=0}^{\infty} |a_n|$ first
\end{itemize}
\end{tcolorbox}

\begin{tcolorbox}[title=WHEN TO USE EACH TEST, colback=white, colframe=red!75!white]
    \underline{The short answer}: ratio/root test is usually the best test to use \textbf{unless:} 1. your $a_n$ is a polynomial or 2. you have one of the special series ($p$-series, geometric, alternating, or telescoping). \\
    \underline{The long answer}: most questions won't really ask you to use a specific test so which test you use is up to you. \textit{However}, with that being said, oftentimes a certain test is easier than another. As stated above, ratio/root test \textit{tends} to be easiest, however, do not under estimate the usefulness of the other tests. For example, a series that is extremely similar to a $p$-series or a geometric series can easily have its convergence be found with a quick comparison test. For a more in depth list of examples see \href{https://tutorial.math.lamar.edu/Classes/CalcII/SeriesStrategy.aspx}{this link}. 
\end{tcolorbox}

\begin{tcolorbox}[title=PRACTICE, colback=white, colframe=blue!75!black]
    As you've heard in many prior units, the best way to learn which series to use when is through lots of practice. So, below are quick links to \href{https://tutorial.math.lamar.edu}{Pauls Online Notes} practice problems for the Sequence's and Series Unit. 
    \begin{itemize}
        \item \href{https://tutorial.math.lamar.edu/Problems/CalcII/IntegralTest.aspx}{Integral Test Practice}
        \item \href{https://tutorial.math.lamar.edu/Problems/CalcII/SeriesCompTest.aspx}{Comparison Test}
        \item \href{https://tutorial.math.lamar.edu/Problems/CalcII/AlternatingSeries.aspx}{Alternating Series Test}
        \item \href{https://tutorial.math.lamar.edu/Problems/CalcII/AbsoluteConvergence.aspx}{Absolute Convergence}
    \end{itemize}
\end{tcolorbox}
\end{document}