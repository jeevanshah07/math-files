\documentclass[addpoints]{exam}
\usepackage{longtable}
\usepackage{array}
\usepackage{amsmath}
\usepackage{amsfonts}
\usepackage{amssymb}
\usepackage[most]{tcolorbox}
\usepackage{tikz}
\usepackage{pgfplots}
\usepackage{mdframed}
\usepackage{hyperref}
\usepackage{amsthm}
\usepackage[x11names, svgnames]{xcolor}
\usepackage{cancel}
\usepackage{tocbibind}

\usetikzlibrary{decorations.markings}

\marksnotpoints
\printanswers
\pointsinrightmargin
\bracketedpoints

\hypersetup{
  colorlinks=true,
  linkcolor=blue,
  linktoc=subsection,
  filecolor=magenta,
  urlcolor=blue,
  pdfpagemode=FullScreen,
}
\urlstyle{same}

\theoremstyle{definition}
\newtheorem*{definition}{Definition}

\theoremstyle{plain}
\renewcommand\qedsymbol{$\blacksquare$}
\newtheorem{theorem}{Theorem}

\renewcommand{\implies}{\Rightarrow}

\newcolumntype{C}[1]{>{\centering\let\newline\\\arraybackslash\hspace{0pt}}m{#1}}


\pagestyle{headandfoot}
\firstpageheadrule
\runningheadrule
\firstpageheader{Calc 2}{Intro to Calculus 2}{Shah}
\runningheader{Intro to Calculus 2}{}{Shah}
\firstpagefooter{}{}{}
\runningfooter{ }{\thepage}{ }

\begin{document}
\section{Notes: Advanced Integration}
\begin{tcolorbox}[breakable, title=\subsection{Integration By Parts}, colframe=black, sharp corners, colback=Azure4!70, colbacktitle=DodgerBlue3!60, coltitle=black]
    Back in calc 1 you learned about how to integrate to undo \textbf{most} of the derivative rules: chain rule, quotient rule, power rule, etc. However, the one rule you didn't learn how to reverse was the \underline{product rule}. In order to undo the product rule we use a process called \emph{integration by parts}. 
    \[
        \int\,u\mathrm{dv} = uv - \int\,v\mathrm{d}u
    \]
    Now naturally you may be wondering what part of the integral should be $u$ and which should be $\mathrm{d}v$. For this we use the acronym LIPET which tells us the priority for $u$:
    \begin{enumerate}
        \item \textbf{L}: logs 
        \item \textbf{I}: inverse trig
        \item \textbf{P}: polynomials
        \item \textbf{E}: exponentials
        \item \textbf{T}: trigs
    \end{enumerate}
\end{tcolorbox}
Evaluate the following integrals.
\begin{questions}
    \question $\displaystyle\,\int\,x\sin\left(2x\right)\,\mathrm{d}x$
    \begin{solution}[\stretch{1}]
        The first step in evaluating any integral is to decide what technique needs to be used. Now, upon inspecting the integral we can see that we've got an $x$ in front of the sine. So, since we've got the product of two functions as our integrand (with no viable $u$-sub) it seems that integration by parts is the way to go. Now, we must choose values for $u$ and $dv$. Going by LIPET, polynomials come before trigs so it seems that choosing $x$ would be a good option for $u$. Besides LIPET, we can also notice that whatever we choose for $u$ is going to be differentiated, and differentiating $x$ will just cause it to drop out, hopefully making our lives easier. So,
        \begin{align*}
            u &= x &\hspace{0.2in} \mathrm{d}v &= \sin\left(2x\right) \\
            \mathrm{d}u &= \mathrm{d}x &\hspace{0.2in} v &= \int\,\sin\left(2x\right)\,\mathrm{d}x = -\frac{1}{2}\cos\left(2x\right)
        \end{align*}
        Now substituting this into our integration by parts formula gives
        \[
            \int\,x\sin\left(2x\right)dx = -\frac{x}{2}\cos\left(2x\right) - \int\,-\frac{1}{2}\cos\left(2x\right)dx = -\frac{x}{2}\cos\left(2x\right) + \frac{1}{4}\sin\left(x\right) + C
        \]
    \end{solution}

    \ifprintanswers
        \newpage
    \fi

    \question $\displaystyle\,\int \arctan{x}\,\mathrm{d}x$
    \begin{solution}[\stretch{1}]
        Having gone over the previous example, less detail will be put into the explanation of this solution. At first glace, this may not seem like an integration by parts integral, however, there is simply no other method that would get us anywhere. Going through LIPET tells us that $\arctan$ would make for a great $u$ but then what could be $\mathrm{d}v$? How about $\mathrm{d}x$?
        \begin{align*}
            u &= \arctan{x} &\hspace{0.2in} \mathrm{d}v &= \mathrm{d}x \\
            \mathrm{d}u &= \frac{1}{{1+x^2}}\mathrm{d}x &\hspace{0.2in} v &= \int\,\mathrm{d}x = x
        \end{align*}
        This gives,
        \[
            \int\,\arctan{x}\,\mathrm{d}x = x\arctan{x} - \int \frac{x}{{1+x^2}}\,\mathrm{d}x = x\arctan{x} - \frac{1}{2}\ln\left|1+x^2\right| + C
        \]
    \end{solution}

    \ifprintanswers
    \else
        \newpage
    \fi

    \question $\displaystyle \int\,x^{2}\sin\left(10x\right)\,\mathrm{d}x$
    \begin{solution}[\stretch{1}]
        Although this is very similar to the first problem, note the difference in the square in the $x$. Through LIPET we can find:
        \begin{align*}
            u &= x^2 &\hspace{0.2in} \mathrm{d}v &= \sin\left(10x\right) \\
            \mathrm{d}u &= 2x\mathrm{d}x &\hspace{0.2in} v &= \int\,\sin\left(10x\right) = -\frac{1}{10}\cos\left(10x\right)
        \end{align*}
        \[
            \Rightarrow \int\,x^2\sin\left(10x\right)\,\mathrm{d}x = -\frac{x^2}{10}\cos\left(10x\right) - \int\,-\frac{2x}{10}\cos\left(10x\right)\,\mathrm{d}x = -\frac{x^2}{10}\cos\left(10x\right) + \frac{1}{5}\int\,x\cos\left(10x\right)\,\mathrm{d}x
        \]
        Now, some pattern recognition of our new integral to evaluate would tell us that we need to apply integration by parts \textbf{again} - something that's completely normal and bound to happen! With $s$ in place of $u$ and $t$ as $v$ we have (for our new integral)
        \begin{align*}
            s &= x &\hspace{0.2in} \mathrm{d}t &= \cos\left(10x\right) \\
            \mathrm{d}s &= \mathrm{d}x &\hspace{0.2in} t &= \int\,\cos\left(10x\right) = \frac{1}{10}\sin\left(10x\right)
        \end{align*}
        \[
            \Rightarrow \int\,x\cos\left(10x\right)\,\mathrm{d}x = \frac{x}{10}\sin\left(10x\right) - \int\,\frac{1}{10}\sin\left(10x\right)\,\mathrm{d}x = \frac{x}{10}\sin\left(10x\right) + \frac{1}{100}\cos\left(10x\right) + c
        \]
        Putting it all together gives,
        \begin{align*}
        &\int\,x^{2}\sin\left(10x\right)\,\mathrm{d}x = -\frac{x^2}{10}\cos\left(10x\right) + \frac{1}{5}\left(\frac{x}{10}\sin\left(10x\right) + \frac{1}{100}\cos\left(10x\right) + c\right) \\ &= \boxed{-\frac{x^2}{10}\cos\left(10x\right) + \frac{x}{50}\sin\left(10x\right) + \frac{1}{500}\cos\left(10x\right) + C}
        \end{align*}
    \end{solution}
\end{questions}

\ifprintanswers 
    \newpage 
\fi


\begin{tcolorbox}[breakable, title=\subsubsection{TABULAR METHOD}, colframe=black, sharp corners, colback=Azure4!30, colbacktitle=Firebrick2!60, coltitle=black]
    As you just saw, sometimes we have to apply integration by parts twice (or occasionally more) in order to get our final answer. One way we can shorten this process is through the tabular method. We begin as usual, choosing our $u$ and $\mathrm{d}v$ but rather than computing $\mathrm{d}u$ and $v$ we use the following table
   \begin{longtable}[h]{|C{1cm}|C{2cm}|C{1cm}|}
    \hline $u$ & $\mathrm{d}v$ & sign  \\\hline
        $\mathrm{d}u $ & $\displaystyle\int\,\mathrm{d}v$ & + \\\hline
        $\mathrm{d}^2u$ & $\displaystyle\int\int\,\mathrm{d}^2v$ & - \\\hline
        $\mathrm{d}^3u $ &  $\displaystyle\int\int\int\,\mathrm{d}^3v$ & + \\\hline
        $\vdots$ &  $\vdots$ & $\vdots$ \\\hline
        $ 0 $ &  $\displaystyle^{n}\int\int\int\,\mathrm{d}^{n}v$ & + \\\hline
    \end{longtable} 
    Essentially, we differentiate $u$ until we hit $0$ and integrate $\mathrm{d}v$ for each corresponding row. The sign column alternates from positive to negative each time. Once we've got our table filled out, we can multiply across the columns to get our answer. 
\end{tcolorbox}
Evaluate the following integral using the tabular method of integration by parts
\begin{questions}
    \question $\displaystyle \int\,x^4e^{\frac{x}{2}}\,\mathrm{d}x$
    \begin{solution}[\stretch{0.5}]
        For this we have the following table 
        \[
\begin{array}{|c|c|c|}
\hline
\begin{array}{c}
u \\ [1em]
x^4 \\[1em]
4x^3 \\[1em]
12x^2 \\[1em]
24x \\[1em]
24 \\[1em]
0
\end{array} &
\begin{array}{c}
\mathrm{d}v \\[1em]
e^{\frac{x}{2}} \\[1em]
2e^{\frac{x}{2}} \\[1em]
4e^{\frac{x}{2}} \\[1em]
8e^{\frac{x}{2}} \\[1em]
16e^{\frac{x}{2}} \\[1em]
32e^{\frac{x}{2}}
\end{array} &
\begin{array}{c}
\text{sign} \\[1em]
+ \\[1em]
- \\[1em]
+ \\[1em]
- \\[1em]
+ \\[1em]
-
\end{array} \\
\hline
\end{array}
\]

    Multiplying diagonally gives
    \begin{align*}\int{{{x^4}{{\bf{e}}^{\frac{x}{2}}}\,dx}} & = \left( {{x^4}} \right)\left( {2{{\bf{e}}^{\frac{x}{2}}}} \right) - \left( {4{x^3}} \right)\left( {4{{\bf{e}}^{\frac{x}{2}}}} \right) + \left( {12{x^2}} \right)\left( {8{{\bf{e}}^{\frac{x}{2}}}} \right) - \left( {24x} \right)\left( {16{{\bf{e}}^{\frac{x}{2}}}} \right) + \left( {24} \right)\left( {32{{\bf{e}}^{\frac{x}{2}}}} \right)\\ &  = 2{x^4}{{\bf{e}}^{\frac{x}{2}}} - 16{x^3}{{\bf{e}}^{\frac{x}{2}}} + 96{x^2}{{\bf{e}}^{\frac{x}{2}}} - 384x{{\bf{e}}^{\frac{x}{2}}} + 768{{\bf{e}}^{\frac{x}{2}}} + c\end{align*}
    \end{solution}
\end{questions}

\newpage 

\begin{tcolorbox}[breakable, title=\subsection{PARTIAL FRACTION DECOMPOSITION}, colframe=black, sharp corners, colback=Azure4!40, colbacktitle=DeepPink2!60, coltitle=black]
    For integrals in the form $\displaystyle\,\int\,\frac{P(x)}{Q(x)}\,dx$ where $P(x)$ and $Q(x)$ are polynomials such that the degree of $P$ is less than the degree of $Q$ and $P$ is \underline{not} the derivative of $Q$, we can perform partial fraction decomposition (PFD) to split our rational function into multiple different functions to form an easier integral. 
    \begin{tcolorbox}[breakable, title=\subsubsection{STEPS FOR PFD}, colframe=black, sharp corners, colback=Azure4!70, colbacktitle=DeepPink1!40, coltitle=black]
        \begin{enumerate}
            \item Ensure that the degree of $P$ is less than the degree of $Q$. \textbf{If it isn't, you must do polynomial long division}
            \item Factor the denominator as completely as possible
            \item Split the denominator into multiple fractions according to the table below
            \item Solve for the coefficients using either a system or a matrix
            \item Rewrite your fraction and integrate
        \end{enumerate}
    \end{tcolorbox}
    \begin{tabular}{c|c}
        \begin{tabular}{c} 
            \textbf{Factor in} \\
            \textbf{denominator}
            \end{tabular} & \begin{tabular}{c} 
            \textbf{Term in partial} \\
            \textbf{fraction decomposition}
            \end{tabular} \\
        \hline
        \( ax+b \) & \( \frac{A}{ax+b} \) \\
        \( (ax+b)^{k} \) & \( \frac{A_{1}}{ax+b}+\frac{A_{2}}{(ax+b)^{2}}+\cdots+\frac{A_{k}}{(ax+b)^{k}}, \, k=1,2,3, \ldots \) \\
        \( ax^{2}+bx+c \) & \( \frac{Ax+B}{ax^{2}+bx+c} \) \\
        \( \left(ax^{2}+bx+c\right)^{k} \) & \( \frac{A_{1}x+B_{1}}{ax^{2}+bx+c}+\frac{A_{2}x+B_{2}}{\left(ax^{2}+bx+c\right)^{2}}+\cdots+\frac{A_{k}x+B_{k}}{\left(ax^{2}+bx+c\right)^{k}}, \, k=1,2,3, \ldots \)
    \end{tabular}
\end{tcolorbox}

Evaluate the following integrals.
\begin{questions}
    \question $\displaystyle\,\int\frac{3x+11}{x^2-x-6}$
    \begin{solution}[\stretch{1}]
        \begin{align*}
            \textbf{\underline{PFD:}} \hspace{0.1in} &\frac{3x+11}{x^2-x-6} = \frac{3x+11}{\left(x-3\right)\left(x+2\right)} = \frac{A}{x-3} + \frac{B}{x+2} \\
            &\Rightarrow 3x+11 = A\left(x+2\right) + B\left(x-3\right)
        \end{align*}
        
        From here there are two difference ways to solve for the coefficients
        
        \begin{minipage}{0.45\linewidth}
            \begin{align*}
                &\Rightarrow 3x+11 = Ax + Bx + 2A - 3B \\
                &\Rightarrow 3x+11 = x(A+B) + (2A - 3B) \\
                &\Rightarrow \begin{cases}
                    A + B &= -3 \\
                    2A - 3B &= 11
                \end{cases}
                \Rightarrow A=4, B=-1
            \end{align*}
        \end{minipage}
        \hfill
        \begin{minipage}{0.45\linewidth}
            We can start by noticing that if $x=-2$ our $A$ term will drop out and if $x=3$ our $B$ term will drop out. 
            \begin{align*}x & =  - 2 : & \hspace{0.05in}5 & = A\left( 0 \right) + B\left( { - 5} \right) & \hspace{0.05in}  & \Rightarrow & \hspace{0.05in}B & =  - 1\\ x & = 3 \,\,\,\,: & \hspace{0.05in}20 & = A\left( 5 \right) + B\left( 0 \right) & \hspace{0.05in} & \Rightarrow & \hspace{0.05in}A & = 4\end{align*}
        \end{minipage}
        \vspace{0.1in}
        \newline
        From here we can finish our PFD and integrate as follows:
        \begin{align*}
            \int\,\frac{3x+11}{x^2-x-6}\,dx &= \int\,\frac{4}{x-3} + \frac{-1}{x+2}\,dx \\
            &= \boxed{4\ln\left|x-3\right| - \ln\left|x+2\right| + C}
        \end{align*}
    \end{solution}

    \newpage

    \question $\displaystyle\,\int\,\frac{x^2-29x+5}{\left(x-4\right)^{2}\left(x^2+3\right)}\,dx$
    \begin{solution}[\stretch{1}]
        \begin{align*}
            & \frac{x^2-29x+5}{\left(x-4\right)^{2}\left(x^2+3\right)} = \frac{A}{x-4} + \frac{B}{\left(x-4\right)^{2}} + \frac{Cx + D}{x^2 + 3} \\
            &\Rightarrow {x^2} - 29x + 5 = A\left( {x - 4} \right)\left( {{x^2} + 3} \right) + B\left( {{x^2} + 3} \right) + \left( {Cx + D} \right){\left( {x - 4} \right)^2} \\
            &\Rightarrow {x^2} - 29x + 5 = \left( {A + C} \right){x^3} + \left( { - 4A + B - 8C + D} \right){x^2} + \left( {3A + 16C - 8D} \right)x - 12A + 3B + 16D \\
        \end{align*}
        This gives the following system of equations which we can represent as a matrix, $M$:
        \begin{equation*}
            \left.
            \begin{split}
                A + C &= 0 \\
                -4A + B - 8C + D &= 1 \\
                3A + 16C - 8D &= -29 \\
                -12A + 3B + 16D &= 5
            \end{split}
            \right\} \Rightarrow
            M = \left[ 
            \begin{array}{ccccc}
                1 & 0 & 1 & 0 & 0  \\
                -4 & 1 & -8 & 1 & 1  \\
                3 & 0 & 16 & -8 & -29  \\
                -12 & 3 & 0 & 16 & 5  \\
            \end{array} 
            \right]
        \end{equation*}
        Now, we can take this matrix in our TI-84 and compute the values for our coefficents using \texttt{rref(M)} (\texttt{rref} is found in 2$^{\text{nd}} \rightarrow \text{x}^{-1} \rightarrow \text{math} \rightarrow \text{B}$). Using \texttt{rref(M)} outputs the following matrix:
        \begin{equation*}
            \left[
            \begin{array}{ccccc}
                1 & 0 & 0 & 0 & 1  \\
                0 & 1 & 0 & 0 & -5  \\
                0 & 0 & 1 & 0 & -1  \\
                0 & 0 & 0 & 1 & 2  \\
            \end{array}
            \right]
        \end{equation*}
        This tells us that, from $A$ to $D$ the coefficients are $1$, $-5$, $-1$, and $2$, respectively. So,
        \begin{align*}\int{{\frac{{{x^2} - 29x + 5}}{{{{\left( {x - 4} \right)}^2}\left( {{x^2} + 3} \right)}}\,dx}} & = \int{{\frac{1}{{x - 4}} - \frac{5}{{{{\left( {x - 4} \right)}^2}}} + \frac{{ - x + 2}}{{{x^2} + 3}}\,dx}}\\ &  = \int{{\frac{1}{{x - 4}} - \frac{5}{{{{\left( {x - 4} \right)}^2}}} - \frac{x}{{{x^2} + 3}}\, + \frac{2}{{{x^2} + 3}}\,dx}}\\ &  = \ln \left| {x - 4} \right| + \frac{5}{{x - 4}} - \frac{1}{2}\ln \left| {{x^2} + 3} \right| + \frac{2}{{\sqrt 3 }}{\tan ^{ - 1}}\left( {\frac{x}{{\sqrt 3 }}} \right) + c\end{align*}
    \end{solution}
\end{questions}
\end{document}