\documentclass[addpoints]{exam}
\usepackage{amsmath}
\usepackage{amsfonts}
\usepackage{amssymb}
\usepackage[most]{tcolorbox}
\usepackage{tikz}
\usepackage{pgfplots}
\usepackage{mdframed}
\usepackage{hyperref}
\usepackage{amsthm}
\usepackage[x11names]{xcolor}
\usepackage{cancel}

\usetikzlibrary{decorations.markings}

\marksnotpoints
\pointsinrightmargin
\bracketedpoints
% \printanswers

\hypersetup{
  colorlinks=true,
  linkcolor=blue,
  filecolor=magenta,
  urlcolor=blue,
  pdfpagemode=FullScreen,
}

\theoremstyle{definition}
\newtheorem*{definition}{Definition}

\theoremstyle{plain}
\renewcommand\qedsymbol{$\blacksquare$}
\newtheorem{theorem}{Theorem}

\renewcommand{\implies}{\Rightarrow}

\urlstyle{same}

\pagestyle{headandfoot}
\firstpageheadrule
\runningheadrule
\firstpageheader{AP Calc BC}{Integration}{Shah}
\runningheader{AP Calc BC}{Integration Notes}{Shah}
\firstpagefooter{}{}{}
\runningfooter{ }{\thepage}{ }

\begin{document}
\section{Notes}
At its most basic level, an introductory look to integrals is that they are the opposite of derivatives. Thus, to integrate a function we must follow the opposite steps to deriving...
\begin{tcolorbox}[breakable, title=\subsection{POWER RULE FOR INTEGRALS}, colframe=black, sharp corners, colback=Azure4!30, colbacktitle=DodgerBlue3!60, coltitle=black]
    Recall that the power rule for \textbf{derivatives} is 
    \[
        \frac{\mathrm{d}}{\mathrm{d}x}\left[x^{n}\right] = nx^{n-1}
    \]
    or, in words: "multiply by the power then decrease the power by one". So, to reverse this, we will negate the statement by flipping its order and doing the opposite to give us (again in words): "increase the power by one then divide by the new power". Symbolically:
    \[
        \int x^{n}\,\mathrm{d}x = \frac{x^{n+1}}{n+1} + c
    \]
    \emph{Note that the power rule does \textbf{not} hold for $n=-1$}
    \begin{tcolorbox}[breakable, title=\subsubsection{CONSTANT OF INTEGRATION}, colframe=black, sharp corners, colback=Azure4!30, colbacktitle=Magenta3!80, coltitle=black]
        As you saw above, the answer to our integral included the letter $c$, why? First, again recall that the derivative of any constant is $0$. Now, again following the idea that integrals are the opposite of derivatives, when we integrate we have to reverse the derivative process. A consequence of that is that we have to account for all the possible constants that could've been in our original equation. Thus, we add a $+c$ to end of every \emph{indefinite} integral.
    \end{tcolorbox}
\end{tcolorbox}
Evaluate the following integrals using the \textbf{power rule for integrals}
\begin{questions}
    \question $\displaystyle\,\int\,x\,\mathrm{d}x$
    \vspace{\stretch{1}}
    \question $\displaystyle\,\int\,4x^3\,\mathrm{d}x$
    \vspace{\stretch{1}}
    \question $\displaystyle\,\int\,5x + 7x^3 + 8\,\mathrm{d}x$
    \vspace{\stretch{1}}
\end{questions}

\newpage 

\begin{tcolorbox}[breakable, title=\subsection{INTEGRATION FORMULAS}, colframe=black, sharp corners, colback=Azure4!30, colbacktitle=Firebrick2!60, coltitle=black]
    Below are the formulas you \textbf{must} memorize
    \begin{enumerate}
        \begin{minipage}{0.45\linewidth}
            \item $\displaystyle\,\int\,\sin\left(x\right)\,\mathrm{d}x = -\cos\left(x\right) + c$
            \item $\displaystyle\,\int\,\cos\left(x\right)\,\mathrm{d}x = \sin\left(x\right) + c$
            \item $\displaystyle\,\int\,\tan\left(x\right)\sec\left(x\right)\left(x\right)\,\mathrm{d}x = \sec\left(x\right) + c$
            \item $\displaystyle\,\int\,\sec^{2}\left(x\right)\,\mathrm{d}x = \tan\left(x\right) + c$
            \item $\displaystyle\,\int\,-\csc\left(x\right)\cot\left(x\right)\,\mathrm{d}x = \csc\left(x\right) + c$
            \item $\displaystyle\,\int\,-\csc^{2}\left(x\right)\,\mathrm{d}x = \cot\left(x\right) + c$
            \item $\displaystyle\,\int\,-\sec\left(x\right)\,\mathrm{d}x = \ln\left|\sec\left(x\right) + \tan\left(x\right)\right| + c$
        \end{minipage}
        \hfill 
        \begin{minipage}{0.45\linewidth}
            \item $\displaystyle\,\int\,\csc\left(x\right)\,\mathrm{d}x = -\ln\left|\csc\left(x\right) + \cot\left(x\right)\right| + c$
            \item $\displaystyle\,\int\,\frac{1}{x}\,\mathrm{d}x = \ln\left|x\right| + c$
            \item $\displaystyle\,\int\,e^{x}\,\mathrm{d}x = e^{x} + c$
            \item $\displaystyle\,\int\,a^{x}\,\mathrm{d}x = \frac{a^{x}}{\ln a} + c$
            \item $\displaystyle\,\int\,\frac{1}{\sqrt{a^2-x^2}}\,\mathrm{d}x = \sin^{-1}\left(\frac{x}{a}\right) + c$
            \item $\displaystyle\,\int\,\frac{1}{{a^2-x^2}}\,\mathrm{d}x = \frac{1}{a}\tan^{-1}\left(\frac{x}{a}\right) + c$
            \item $\displaystyle\,\int\,\frac{1}{\left|x\right|\sqrt{x^2-a^2}}\,\mathrm{d}x = \frac{1}{a}\sec^{-1}\left(\frac{x}{a}\right) + c$
        \end{minipage}
    \end{enumerate}
\end{tcolorbox}
Compute the following integrals using the formulas above
\begin{questions}
    \question $\displaystyle\,\int\,\sin\left(x\right)\,\mathrm{d}x$
    \vspace{\stretch{.5}}
    
    \question $\displaystyle\,\int\,\frac{4}{x}\,\mathrm{d}x$
    \vspace{\stretch{.5}}
    
    \question $\displaystyle\,\int\,\frac{1}{\sqrt{16-x^2}}\,\mathrm{d}x$
    \vspace{\stretch{0.5}}
    
    \question $\displaystyle\,\int\,\frac{1}{{16-x^2}}\,\mathrm{d}x$
    \vspace{\stretch{0.5}}
\end{questions}

\newpage 

\begin{tcolorbox}[breakable, title=\subsection{U-SUB}, colframe=black, sharp corners, colback=Azure4!30, colbacktitle=DeepPink2!60, coltitle=black]
    In our journey to learn how to integrate all different types of functions we are bound to come across compositions of functions. The motivation behind the $u$-sub technique is to undo the chain rule, the technique used to \emph{derive} compositions of functions. Lets start by defining a function $F(x)=f(g(x))$ and recalling that the chain rule tells us that $F'(x)=f'(g(x))g'(x)$. Now, should we want to back to $F$ from $F'$ we just need to compute the following integral
    \[
        F(x) = \int\,F'(x)\,\mathrm{d}x = \int\,f'(g(x))g'(x)\,\mathrm{d}x
    \]
    However, this integral is almost always never in a form that we can integrate using our known formulas (imagine something like $\displaystyle\,\int\,\left(8x-1\right)e^{4x^2-x}\,\mathrm{d}x$), so we use the idea of $u$-sub to transform our integral into one that we can easily evaluate using known formulas. The steps for $u$-sub are as follows
    \begin{enumerate}
        \item Let $u=g(x)$
        \item Compute a differential for your substitution $\mathrm{d}u=g'(x)\mathrm{d}x$ (all this is, is taking the derivative of each side with respect to the appropriate variable)
        \item Use your substitution and following differential in your integral: $\displaystyle\,\int\,f(g'(x))g'(x)\,mathrm{d}x\Rightarrow\int\,f(u)\,\mathrm{d}u$ {\small (note that your differential will not always appear exactly in the integral, it may take some manipulate to make the substitution work\footnote{for help see the properties section})}
        \item Integrate with respect to $u$: $\displaystyle\,\int\,f(u)\,\mathrm{d}u = F(u) + c$
        \item Substitute back in from your substitution: $F(u)\Rightarrow\,F(g(x))$ (because $u=g(x)$)
        \item Finally, put it all together to get $\displaystyle\,\int\,F'(x)\,\mathrm{d}x = F(g(x)) + c$
    \end{enumerate}
\end{tcolorbox}
Compute the following integrals using \textbf{$u$-sub}
\begin{questions}
    \begin{minipage}{0.45\linewidth}
        \question $\displaystyle\,\int\,x^2\left(3-10x^3\right)^{4}\,\mathrm{d}x$ 
    \end{minipage}
    \hfill
    \begin{minipage}{0.45\linewidth}
        \question $\displaystyle\,\int\,3\left(8x-1\right)e^{4x^2-x}\,\mathrm{d}x$
    \end{minipage}
    \vspace{\stretch{0.7}}
    \question $\displaystyle\,\int\,\cos\left(3x\right)\sin^{10}\left(3x\right)\,\mathrm{d}x$
    \vspace{\stretch{0.5}}
\end{questions}

\newpage 

\begin{tcolorbox}[breakable, title=\subsection{DEFINITE INTEGRALS}, colframe=black, sharp corners, colback=Azure4!30, colbacktitle=DarkOrchid2!60, coltitle=black]
    Up to this point we've so far only been computing \textbf{indefinite} integrals, but there's a second type of integral: the \textbf{definite} integral. A definite integral is an integrals with bounds, $a$ and $b$, and looks like $\displaystyle\,\int_{a}^{b}\,F(x)\,\mathrm{d}x$. Luckily, to compute a \emph{definite} integrals, all you need to do is compute the \emph{indefinite} integral and then numerically evaluate for your bounds. If $f$ is a function and $f'$ is it's derivative then
    \[
        \int_{a}^{b}\,f'(x)\,\mathrm{d}x = f(b) - f(a)
    \]
    You'll also note that there is \underline{no constant of integration when evaluate a definite integral}
\end{tcolorbox}
Compute the following \textbf{definite integrals}
\begin{questions}
    \begin{minipage}{0.45\linewidth}
        \question $\displaystyle\,\int_{0}^{1}\,x^2\left(3-10x^3\right)^{4}\,\mathrm{d}x$ 
    \end{minipage}
    \hfill
    \begin{minipage}{0.45\linewidth}
        \question $\displaystyle\,\int_{0}^{6}\,3\left(8x-1\right)e^{4x^2-x}\,\mathrm{d}x$
    \end{minipage}
    \vspace{\stretch{0.7}}
    \question $\displaystyle\int_{2}^{3}\,x^2-3\,\mathrm{d}x$
    \vspace{\stretch{0.5}}
\end{questions}

\newpage 
\begin{tcolorbox}[breakable, title=\subsection{BACK SUB}, colframe=black, sharp corners, colback=Azure4!30, colbacktitle=Plum2!80, coltitle=black]
    Back-sub is a technique that expands upon $u$-sub to allow us to integrate more types of functions. Take the integral of $x\sqrt{x-5}$. We can't multiply the $x$ into the root, and if we try to use $u=x-5$ our differential does no help to us. However, we can use our substitution to rewrite $x$ in terms of $u$: $u=x-5\Rightarrow\,x=u+5$. Now, using the fact that $\mathrm{d}u=\mathrm{d}x$ we can rewrite the integral as follows
    \[
        \int\,x\sqrt{x-5}\mathrm{d}x = \int\,\left(u+5\right)u^{1/2}\,\mathrm{d}u = \int\,u^{3/2}+5u^{1/2}\,\mathrm{d}u
    \]
    Now, the final form of the integral is one we can easily integrate to find our anti-derivative. After integrating, all that's left is to substitute back in for $x$ and add our constant of integration. 
\end{tcolorbox}
Evaluate the following integrals
\begin{questions}
    \question $\displaystyle\,\int\,x^2\left(2x+3\right)^{3}\,\mathrm{d}x$
    \vspace{\stretch{1}}
    
    \question $\displaystyle\,\int\,\frac{x^2}{x+1}\,\mathrm{d}x$
    \vspace{\stretch{1}}
    
    \question $\displaystyle\,\int\,x\sqrt{x^2-3}\,\mathrm{d}x$
    \vspace{\stretch{1}}
\end{questions}

\newpage 

\begin{tcolorbox}[breakable, title=\subsection{EXTRAS}, colframe=black, sharp corners, colback=Azure4!70, colbacktitle=SlateBlue3!60, coltitle=black]
    \begin{tcolorbox}[breakable, title=\subsubsection{PROPERTIES OF INTEGRALS}, colframe=black, sharp corners, colback=Azure4!30, colbacktitle=SeaGreen3!80, coltitle=black]
        Note that the following properties hold for indefinite \textbf{and} definite integrals
        \begin{enumerate}
            \item $\displaystyle\,\int\,kf(x)\,\mathrm{d}x = k\int\,f(x)\,\mathrm{d}x \hspace{0.15in} \text{where } k\in\mathbb{R}$
            \item $\displaystyle\,\int\,-f(x)\,\mathrm{d}x = -\int\,f(x)\,\mathrm{d}x$ {\small (This is really just property 1 with $k=-1$}
            \item $\displaystyle\,\int\,f(x)\pm\,g(x)\,\mathrm{d}x = \int\,f(x)\,\mathrm{d}x\pm\,\int\,g(x)\,\mathrm{d}x$
        \end{enumerate}
    \end{tcolorbox}
    \begin{tcolorbox}[breakable, title=\subsubsection{PROPERTIES OF DEFINITE INTEGRALS}, colframe=black, sharp corners, colback=Azure4!30, colbacktitle=Magenta3!80, coltitle=black]
        Note that these are properite of \textbf{only} definite integrals
        \begin{enumerate}
            \item $\displaystyle\,\int_{a}^{b}\,f(x)\,\mathrm{d}x = -\int_{b}^{a}\,f(x)\,\mathrm{d}x$
            \item $\displaystyle\,\int_{a}^{b}\,f(x)\,\mathrm{d}x = 0$
            \item $\displaystyle\,\int_{a}^{b}\,f(x)\,\mathrm{d}x = \int_{a}^{c}\,f(x)\,\mathrm{d}x + \int_{c}^{b}\,f(x)\,\mathrm{d}x \hspace{0.15in} \text{where } c\in\mathbb{R}$
        \end{enumerate}
    \end{tcolorbox}
    \begin{tcolorbox}[breakable, title=\subsubsection{PROPERTIES OF EXPONENTS}, colframe=black, sharp corners, colback=Azure4!30, colbacktitle=Orange2!80, coltitle=black]
        \begin{enumerate}
            \item $\displaystyle\,x^{n} \cdot x^{m} = x^{n+m}$
            \item $\displaystyle\,\left(x^{n}\right)^{m} = x^{n\cdot\,m}$
            \item  $\displaystyle\,x^{-n} = \frac{1}{x^n}$
        \end{enumerate}
    \end{tcolorbox}
\end{tcolorbox}

\newpage 

\section{Problem Set}
Complete the following problems. Answers are in the next section. For help consider using \href{https://integral-calculator.net}{\underline{integral calculator}}, SchoolAI, or asking me, \underline{before checking the answer key}!

\begin{questions}
\question Evaluate the following integrals 
    \begin{parts}
    \begin{minipage}{0.45\textwidth}
        \part $\displaystyle \int x (x^2 + 1)^3 \, dx$
        
        \vspace{1em}
        \part $\displaystyle \int \frac{x}{x^2 + 1} \, dx$
        
        \vspace{1em}
        \part $\displaystyle \int (2x + 3)^4 \, dx$
        
        \vspace{1em}
        \part $\displaystyle \int x \sqrt{x^2 + 4} \, dx$
        
        \vspace{1em}
        \part $\displaystyle \int \cos(3x + 2) \, dx$
    \end{minipage}
    \hfill
    \begin{minipage}{0.45\textwidth}
        \part $\displaystyle \int x e^{x^2} \, dx$
        
        \vspace{1em}
        \part $\displaystyle \int \frac{1}{\sqrt{2x + 5}} \, dx$
        
        \vspace{1em}
        \part $\displaystyle \int x \ln(x^2) \, dx$
        
        \vspace{1em}
        \part $\displaystyle \int e^{3x} \, dx$
        
        \vspace{1em}
        \part $\displaystyle \int \sin(2x) \, dx$
    \end{minipage}
\end{parts}
\question Evaluate the following definite integrals
    \begin{parts}
        \begin{minipage}{0.45\textwidth}
            \part $\displaystyle \int_0^2 x (x^2 + 1) \, dx$
            
            \vspace{1em}
            \part $\displaystyle \int_1^3 \frac{x}{x^2 + 1} \, dx$
            
            \vspace{1em}
            \part $\displaystyle \int_{-1}^1 (2x + 3)^2 \, dx$
            
            \vspace{1em}
            \part $\displaystyle \int_0^{\pi/4} \cos(2x) \, dx$
            
            \vspace{1em}
            \part $\displaystyle \int_0^1 \frac{1}{\sqrt{2x + 1}} \, dx$
        \end{minipage}
        \hfill
        \begin{minipage}{0.45\textwidth}
            \part $\displaystyle \int_1^e \frac{\ln x}{x} \, dx$
            
            \vspace{1em}
            \part $\displaystyle \int_0^2 x e^{x^2} \, dx$
            
            \vspace{1em}
            \part $\displaystyle \int_0^{\pi/2} \sin(3x) \, dx$
            
            \vspace{1em}
            \part $\displaystyle \int_0^1 x^2 (x^3 + 1)^4 \, dx$
            
            \vspace{1em}
            \part $\displaystyle \int_1^4 \frac{x}{\sqrt{x^2 + 4}} \, dx$
        \end{minipage} 
    \end{parts}

    \question Evaluate
    \begin{parts}
        \begin{minipage}{0.45\linewidth}
             \part $\displaystyle \int x^3 \sqrt{x - 2} \, dx$
        
            \vspace{1em}
            \part $\displaystyle \int_1^2 \frac{x}{\sqrt{x + 4}} \, dx$
            
            \vspace{1em}
            \part $\displaystyle \int x^2 \sqrt{x + 6} \, dx$
        \end{minipage}
        \hfill
        \begin{minipage}{0.45\linewidth}
            \vspace{1em}
            \part $\displaystyle \int_0^1 x \sqrt{x^2 - 1} \, dx$
            
            \vspace{1em}
            \part $\displaystyle \int x \sqrt{5x - 4} \, dx$

            \vspace{1em}
            \part $\displaystyle\,\int\,\frac{x^3}{x^2 - x}\, dx$
        \end{minipage}
    \end{parts}
\end{questions}

\newpage

\section{Answer Key}
\begin{questions}
\begin{minipage}{0.45\linewidth}
    \question
\begin{parts}
    \part $\displaystyle \frac{(x^2 + 1)^4}{4} + C$
    \part $\displaystyle \frac{1}{2} \ln(x^2 + 1) + C$
    \part $\displaystyle \frac{(2x+3)^5}{10} + C$
    \part $\displaystyle \frac{1}{3}(x^2 + 4)^{3/2} + C$
    \part $\displaystyle \frac{\sin(3x + 2)}{3} + C$
    \part $\displaystyle \frac{1}{2} e^{x^2} + C$
    \part $\displaystyle \sqrt{2x + 5} + C$
    \part $\displaystyle \frac{1}{4} x^2 \ln(x^2) - \frac{1}{8} x^2 + C$
    \part $\displaystyle \frac{e^{3x}}{3} + C$
    \part $\displaystyle -\frac{\cos(2x)}{2} + C$
\end{parts}

\question
\begin{parts}
    \part $\displaystyle \frac{2}{3} \left[(x^2 + 1)^{3/2} \right]_0^2 = \frac{26\sqrt{5} - 2}{3}$
    \part $\displaystyle \frac{1}{2} \left[\ln(x^2 + 1)\right]_1^3 = \frac{1}{2} \ln\left(\frac{10}{2}\right)$
    \part $\displaystyle \frac{(2x + 3)^3}{6} \bigg|_{-1}^1 = \frac{(5)^3 - (1)^3}{6} = \frac{124}{6}$
    \part $\displaystyle \frac{\sin(2x)}{2} \bigg|_0^{\pi/4} = \frac{\sqrt{2} - 1}{2}$
    \part $\displaystyle 2\sqrt{2x+1} \bigg|_0^1 = 2(\sqrt{3} - 1)$
    \part $\displaystyle \frac{(\ln x)^2}{2} \bigg|_1^e = \frac{1}{2}$
    \part $\displaystyle \frac{1}{4} e^{x^2} \bigg|_0^2 = \frac{e^4 - 1}{4}$
    \part $\displaystyle -\frac{\cos(3x)}{3} \bigg|_0^{\pi/2} = \frac{1}{3}$
    \part $\displaystyle \frac{1}{5} (x^3+1)^5 \bigg|_0^1 = \frac{2}{5}$
    \part $\displaystyle \sqrt{x^2 + 4} + C$
\end{parts}
\end{minipage}
\hfill 
\begin{minipage}{0.45\linewidth}
    
\question
\begin{parts}
    \part $\displaystyle \frac{2}{11}(x - 2)^{11/2} + C$
    \part $\displaystyle \frac{2}{3}(x + 4)^{3/2} \bigg|_1^2 = \frac{2}{3}(6^{3/2} - 5^{3/2})$
    \part $\displaystyle \frac{2}{7}(x + 6)^{7/2} + C$
    \part $\displaystyle \frac{1}{3}(x^2 - 1)^{3/2} \bigg|_0^1 = 0$
    \part $\displaystyle \frac{2}{15}(5x - 4)^{5/2} + C$
    \part $\displaystyle\, \frac{x^{2} + 2x}{2} + \ln\left(\left|x - 1\right|\right)$
\end{parts}
\end{minipage}
\end{questions}
\end{document}