\documentclass[addpoints]{exam}
\usepackage{longtable}
\usepackage{array}
\usepackage{amsmath}
\usepackage{amsfonts}
\usepackage{amssymb}
\usepackage[most]{tcolorbox}
\usepackage{tikz}
\usepackage{pgfplots}
\usepackage{mdframed}
\usepackage{hyperref}
\usepackage{amsthm}
\usepackage[x11names, svgnames]{xcolor}
\usepackage{cancel}
\usepackage{tocbibind}

\usetikzlibrary{decorations.markings}

\marksnotpoints
\printanswers
\pointsinrightmargin
\bracketedpoints

\hypersetup{
  colorlinks=true,
  linkcolor=blue,
  linktoc=subsection,
  filecolor=magenta,
  urlcolor=blue,
  pdfpagemode=FullScreen,
}
\urlstyle{same}

\theoremstyle{definition}
\newtheorem*{definition}{Definition}

\theoremstyle{plain}
\renewcommand\qedsymbol{$\blacksquare$}
\newtheorem{theorem}{Theorem}

\renewcommand{\implies}{\Rightarrow}

\newcolumntype{C}[1]{>{\centering\let\newline\\\arraybackslash\hspace{0pt}}m{#1}}


\pagestyle{headandfoot}
\firstpageheadrule
\runningheadrule
\firstpageheader{Calc 2}{Intro to Calculus 2}{Shah}
\runningheader{Intro to Calculus 2}{}{Shah}
\firstpagefooter{}{}{}
\runningfooter{ }{\thepage}{ }

\begin{document}
\section{Notes: Advanced Integration}
\begin{tcolorbox}[breakable, title=\subsection{Integration By Parts}, colframe=black, sharp corners, colback=Azure4!70, colbacktitle=DodgerBlue3!60, coltitle=black]
    Back in calc 1 you learned about how to integrate to undo \textbf{most} of the derivative rules: chain rule, quotient rule, power rule, etc. However, the one rule you didn't learn how to reverse was the \underline{product rule}. In order to undo the product rule we use a process called \emph{integration by parts}. 
    \[
        \int\,u\mathrm{dv} = uv - \int\,v\mathrm{d}u
    \]
    Now naturally you may be wondering what part of the integral should be $u$ and which should be $\mathrm{d}v$. For this we use the acronym LIPET which tells us the priority for $u$:
    \begin{enumerate}
        \item \textbf{L}: logs 
        \item \textbf{I}: inverse trig
        \item \textbf{P}: polynomials
        \item \textbf{E}: exponentials
        \item \textbf{T}: trigs
    \end{enumerate}
\end{tcolorbox}
Evaluate the following integrals.
\begin{questions}
    \question $\displaystyle\,\int\,xe^{6x}\,\mathrm{d}x$
    \begin{solution}[\stretch{1}]
        The first step in evaluating any integral is to decide what technique needs to be used. Now, upon inspecting the integral we can see that we've got an $x$ in front of the exponential. So, since we've got the product of two functions as our integrand (with no viable $u$-sub) it seems that integration by parts is the way to go. Now, we must choose values for $u$ and $dv$. Going by LIPET, polynomials come before exponentials so it seems that choosing $x$ would be a good option for $u$. Besides LIPET, we can also notice that whatever we choose for $u$ is going to be differentiated, and differentiating $x$ will just cause it to drop out, hopefully making our lives easier. So,
        \begin{align*}
            u &= x &\hspace{0.2in} \mathrm{d}v &= e^{6x} \\
            \mathrm{d}u &= \mathrm{d}x &\hspace{0.2in} v &= \int\,e^{6x}\,\mathrm{d}x = \frac{1}{6}e^{6x}
        \end{align*}
        Now substituting this into our integration by parts formula gives
        \[
            \int\,xe^{6x}\,\mathrm{d}x = x\left(\frac{1}{6}e^{6x}\right) - \int\,\frac{1}{6}e^{6x}\,\mathrm{d}x = \frac{x}{6}e^{6x} - \frac{1}{36}e^{6x} + C
        \]
    \end{solution}

    \question $\displaystyle\, \arcsin{x}\,\mathrm{d}x$
    \begin{solution}[\stretch{1}]
        Having gone over the previous example, less detail will be put into the explanation of this solution. At first glace, this may not seem like an integration by parts integral, however, there is simply no other method that would get us anywhere. Going through LIPET tells us that $\arcsin$ would make for a great $u$ but then what could be $\mathrm{d}v$? How about $\mathrm{d}x$?
        \begin{align*}
            u &= \arcsin{x} &\hspace{0.2in} \mathrm{d}v &= \mathrm{d}x \\
            \mathrm{d}u &= \frac{1}{\sqrt{1-x^2}}\mathrm{d}x &\hspace{0.2in} v &= \int\,\mathrm{d}x = x
        \end{align*}
        This gives,
        \[
            \int\,\arcsin{x}\,\mathrm{d}x = x\arcsin{x} - \int \frac{x}{\sqrt{1-x^2}}\,\mathrm{d}x = x\arcsin{x} - \sqrt{1-x^2} + c
        \]
    \end{solution}

    \ifprintanswers
    \else
        \newpage
    \fi

    \question $\displaystyle \int\,x^{2}\sin\left(10x\right)\,\mathrm{d}x$
    \begin{solution}[\stretch{1}]
        Getting right into this we have (through LIPET)
        \begin{align*}
            u &= x^2 &\hspace{0.2in} \mathrm{d}v &= \sin\left(10x\right) \\
            \mathrm{d}u &= 2x\mathrm{d}x &\hspace{0.2in} v &= \int\,\sin\left(10x\right) = -\frac{1}{10}\cos\left(10x\right)
        \end{align*}
        \[
            \Rightarrow \int\,x^2\sin\left(10x\right)\,\mathrm{d}x = -\frac{x^2}{10}\cos\left(10x\right) - \int\,-\frac{2x}{10}\cos\left(10x\right)\,\mathrm{d}x = -\frac{x^2}{10}\cos\left(10x\right) + \frac{1}{5}\int\,x\cos\left(10x\right)\,\mathrm{d}x
        \]
        Now, some pattern recognition of our new integral to evaluate would tell us that we need to apply integration by parts \textbf{again} - something that's completely normal and bound to happen! With $s$ in place of $u$ and $t$ as $v$ we have (for our new integral)
        \begin{align*}
            s &= x &\hspace{0.2in} \mathrm{d}t &= \cos\left(10x\right) \\
            \mathrm{d}s &= \mathrm{d}x &\hspace{0.2in} t &= \int\,\cos\left(10x\right) = \frac{1}{10}\sin\left(10x\right)
        \end{align*}
        \[
            \Rightarrow \int\,x\cos\left(10x\right)\,\mathrm{d}x = \frac{x}{10}\sin\left(10x\right) - \int\,\frac{1}{10}\sin\left(10x\right)\,\mathrm{d}x = \frac{x}{10}\sin\left(10x\right) + \frac{1}{100}\cos\left(10x\right) + c
        \]
        Putting it all together gives,
        \begin{align*}
        &\int\,x^{2}\sin\left(10x\right)\,\mathrm{d}x = -\frac{x^2}{10}\cos\left(10x\right) + \frac{1}{5}\left(\frac{x}{10}\sin\left(10x\right) + \frac{1}{100}\cos\left(10x\right) + c\right) \\ &= \boxed{-\frac{x^2}{10}\cos\left(10x\right) + \frac{x}{50}\sin\left(10x\right) + \frac{1}{500}\cos\left(10x\right) + C}
        \end{align*}
    \end{solution}
\end{questions}

\ifprintanswers 
    \newpage 
\fi


\begin{tcolorbox}[breakable, title=\subsubsection{TABULAR METHOD}, colframe=black, sharp corners, colback=Azure4!30, colbacktitle=Firebrick2!60, coltitle=black]
    As you just saw, sometimes we have to apply integration by parts twice (or occasionally more) in order to get our final answer. One way we can shorten this process is through the tabular method. We begin as usual, choosing our $u$ and $\mathrm{d}v$ but rather than computing $\mathrm{d}u$ and $v$ we use the following table
   \begin{longtable}[h]{|C{1cm}|C{2cm}|C{1cm}|}
    \hline $u$ & $\mathrm{d}v$ & sign  \\\hline
        $\mathrm{d}u $ & $\displaystyle\int\,\mathrm{d}v$ & + \\\hline
        $\mathrm{d}^2u$ & $\displaystyle\int\int\,\mathrm{d}^2v$ & - \\\hline
        $\mathrm{d}^3u $ &  $\displaystyle\int\int\int\,\mathrm{d}^3v$ & + \\\hline
        $\vdots$ &  $\vdots$ & $\vdots$ \\\hline
        $ 0 $ &  $\displaystyle^{n}\int\int\int\,\mathrm{d}^{n}v$ & + \\\hline
    \end{longtable} 
    Essentially, we differentiate $u$ until we hit $0$ and integrate $\mathrm{d}v$ for each corresponding row. The sign column alternates from positive to negative each time. Once we've got our table filled out, we can multiply across the columns to get our answer. 
\end{tcolorbox}
Evaluate the following integral using the tabular method of integration by parts
\begin{questions}
    \question $\displaystyle \int\,x^4e^{\frac{x}{2}}\,\mathrm{d}x$
    \begin{solution}[\stretch{0.5}]
        For this we have the following table 
        \[
\begin{array}{|c|c|c|}
\hline
\begin{array}{c}
u \\ [1em]
x^4 \\[1em]
4x^3 \\[1em]
12x^2 \\[1em]
24x \\[1em]
24 \\[1em]
0
\end{array} &
\begin{array}{c}
\mathrm{d}v \\[1em]
e^{\frac{x}{2}} \\[1em]
2e^{\frac{x}{2}} \\[1em]
4e^{\frac{x}{2}} \\[1em]
8e^{\frac{x}{2}} \\[1em]
16e^{\frac{x}{2}} \\[1em]
32e^{\frac{x}{2}}
\end{array} &
\begin{array}{c}
\text{sign} \\[1em]
+ \\[1em]
- \\[1em]
+ \\[1em]
- \\[1em]
+ \\[1em]
-
\end{array} \\
\hline
\end{array}
\]

    Multiplying diagonally gives
    \begin{align*}\int{{{x^4}{{\bf{e}}^{\frac{x}{2}}}\,dx}} & = \left( {{x^4}} \right)\left( {2{{\bf{e}}^{\frac{x}{2}}}} \right) - \left( {4{x^3}} \right)\left( {4{{\bf{e}}^{\frac{x}{2}}}} \right) + \left( {12{x^2}} \right)\left( {8{{\bf{e}}^{\frac{x}{2}}}} \right) - \left( {24x} \right)\left( {16{{\bf{e}}^{\frac{x}{2}}}} \right) + \left( {24} \right)\left( {32{{\bf{e}}^{\frac{x}{2}}}} \right)\\ &  = 2{x^4}{{\bf{e}}^{\frac{x}{2}}} - 16{x^3}{{\bf{e}}^{\frac{x}{2}}} + 96{x^2}{{\bf{e}}^{\frac{x}{2}}} - 384x{{\bf{e}}^{\frac{x}{2}}} + 768{{\bf{e}}^{\frac{x}{2}}} + c\end{align*}
    \end{solution}
\end{questions}

\newpage 

\begin{tcolorbox}[breakable, title=\subsection{TRIG INTEGRALS}, colframe=black, sharp corners, colback=Azure4!70, colbacktitle=DeepPink2!60, coltitle=black]
Many integrals will at one point come to be some combination of two different trig functions. When dealing with these trig integrals you standardly want to find some way to get them as one factor of cosine and some expression of sine (or vice versa).  
\end{tcolorbox}
Evaluate the following integrals.
\begin{questions}
    \question $\displaystyle\,\int\,\sin^{5}x\,\mathrm{d}x$
    \begin{solution}[\stretch{1}]
        This integral becomes easy work if we recall the pythagorean identity: 
        \[
        \sin^{2}x + \cos^{2}x = 1
        \]
        We can apply this as follows
        \begin{align*}
            \int \sin^{5}x\,\mathrm{d}x &= \int\,\sin^{4}x\left(\sin x\right)\,\mathrm{d}x \\
            &= \int\,\left(\sin^{2}x\right)^2\sin x\,\mathrm{d}x \\
            &= \int\,\left(1-\cos^{2}x\right)^2\sin x\,\mathrm{d}x \\
            &= -\int\,\left(1-u^2\right)^2\,\mathrm{d}u \tag{$u=\cos x\implies \mathrm{d}u = -\sin x\,\mathrm{d}x$} \\
            &= -\int\,1 - 2u^2 + u^4\,\mathrm{d}u \\
            &= -u + \frac{2}{3}u^{3} - \frac{1}{5}u^5 + c \\
            &= -\cos x + \frac{2}{3}\cos^{3}x - \frac{1}{5}\cos^{5}x + c
        \end{align*}
    \end{solution}

    \question $\displaystyle\,\sec^{9}x\tan^{5}x\,\mathrm{d}x$
    \begin{solution}[\stretch{1}]
        Recall that we can derive from the pythagorean identity:
        \[
            \tan^{2}x + 1 = \sec^{2} x
        \]
        This can be applied as follows:
        \begin{align*}\int{{ \sec^9 x \tan^5 x \,dx}} & = \int{{ \sec^8 x \tan^4 x\,\tan x\sec x\,dx}}\\ &  = \int{{{{\sec }^8}x{{\left( {{{\sec }^2}x - 1} \right)}^2}\,\tan x\sec x\,dx}}\tag{$u = \sec x$}\\ &  = \int{{{u^8}{{\left( {{u^2} - 1} \right)}^2}\,du}}\\ &  = \int{{{u^{12}} - 2{u^{10}} + {u^8}\,du}}\\ &  = \frac{1}{{13}}{\sec ^{13}}x - \frac{2}{{11}}{\sec ^{11}}x + \frac{1}{9}{\sec ^9}x + c\end{align*}
    \end{solution}
    
\end{questions}
\end{document}
