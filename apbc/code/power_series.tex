\documentclass[addpoints]{exam}
\usepackage{amsmath}
\usepackage{amsfonts}
\usepackage{amssymb}
\usepackage{tcolorbox}
\usepackage{tikz}
\usepackage{pgfplots}
\usepackage{mdframed}
\usepackage{tcolorbox}


%=============================================
        %POINTS FORMATTING
%=============================================
\marksnotpoints 
%if you would prefer the exam to say "points" instead of "marks", you can delete the above line.

%\pointformat{\textbf{(\thepoints)}} 
%delete comment if you want bold points

\pointsinrightmargin
%\marginpointname{ \points}
\bracketedpoints
%delete comment if you want point values to be printed in the right margin instead of at the start of the question. The same command with the "right" places the point values on the left margin.

%\
%=============================================
        %HEADER/FOOTER FORMATTING
%=============================================
\pagestyle{headandfoot}
\firstpageheadrule
\runningheadrule
\firstpageheader{Unit 7: Power Series}{Notes}{Shah \\ BC Calc}
\runningheader{Power Series}{}{BC Calc}
\firstpagefooter{}{}{}
\runningfooter{ }{\thepage}{ }

\begin{document}

\begin{tcolorbox}[title= RADIUS OF CONVERGENCE,colframe=black,sharp corners,colback=white,colbacktitle=white,coltitle=black]

   \large \textbf{Definition} \\
   \normalsize Every power series has what we call a \textbf{radius of convergence}. This is some number $R$ such that \(\left|x-c\right| < R\), where $R$ is the number for which the power series converges. \\
   \large \textbf{Finding the R.O.C} \\
   \normalsize Lets start by remembering the general form of an infinite power series:
    \[\sum_{n=0}^{\infty}a_n (x-c)^n=a_0 + a_1 (x-c)+a_2 (x-c)^2+a_3 (x-c)^3+\cdots+a_n (x-c)^n+\cdots\]
    Now lets think about how we might find the convergence of this series. We know from previous sections that series with an $n$ in the exponent can easily have their convergence be determined by the ratio and root test. Now lets recall the ratio and root tests:
    Given a series \(\displaystyle
        \sum\limits_{n=k}^{\infty} a_n
    \)
    \[ \displaystyle
        \text{Let } L =
        \overbrace{\lim\limits_{n\to\infty}\left|\frac{a_{n+1}}{a_n}\right|}^{\text{Ratio Test}}
        \text{ or }
        \overbrace{\lim\limits_{n\to\infty}\left|{a_n}\right|^{\frac{1}{n}}}^{\text{Root Test}}
    \]
    \begin{enumerate}
        \item If $L<1$ the series converges
        \item If $L>1$ the series diverges
        \item If $L=1$ the ratio/root tests are inconclusive
    \end{enumerate}
    Now we can apply this to our power series (I will use the ratio test in this example):
    \[
    L = \lim\limits_{n\to\infty}\, \left|\frac{a_{n+1}\left({x-c}\right)^{n+1}}{a_n\left({x-c}\right)^{n}} \right| < 1
    \]
    Note that we create an inequality that sets the value of the limit to be less than 1 since we want the series to converge and we know from above that $L<1$ is the condition required for convergence. \\
    Now we can evaluate this limit:
    \[
    L = \left|x-c\right| \lim\limits_{n\to\infty}\left|\frac{a_{n+1}}{a_n}\right| < 1
    \]
    We can pull the $(x-c)$ out of the limit because the limit does not 'depend' on it, however we must keep the absolute value bars on it since we do not know if it is going to be positive or not.
    At this point we can solve our inequality for \(\left|x-c\right|\) to find our radius of convergence:
    \[
    \left|x-c\right| < \frac{1}{ \lim\limits_{n\to\infty}\left|\frac{a_{n+1}}{a_n}\right|}
    \]
    Now at this point we can notice that we have an inequality in the form \(\left|x-c\right| < R\) where \\ \(R = \frac{1}{ \lim\limits_{n\to\infty}\left|\frac{a_{n+1}}{a_n}\right|}\). Therefore, we can conclude our radius of convergence to be 
    \[
    R = \frac{1}{ \lim\limits_{n\to\infty}\left|\frac{a_{n+1}}{a_n}\right|} 
    \]
    
    
\end{tcolorbox}
\newpage

\begin{tcolorbox}[title= RADIUS OF CONVERGENCE (CONT),colframe=black,sharp corners,colback=white,colbacktitle=white,coltitle=black]
    \large \textbf{Special Cases} \normalsize
    \begin{enumerate}
        \item If the limit goes to infinity then the radius of convergence is 0. (\(R=0\))
        \item If the limit goes to 0 then the radius of convergence is infinity (\(R=\infty\))
    \end{enumerate}
\end{tcolorbox}

Lets practice this idea with some examples. Determine \textbf{only} the radius of convergence of the following series. 
\begin{questions}
    \question \(\displaystyle
    \sum\limits_{n = 1}^\infty  {\frac{{{{\left( { - 1} \right)}^n}n}}{{{4^n}}}{{\left( {x + 3} \right)}^n}}
    \)
    \vspace{\stretch{1}}
    \question \(\displaystyle
   \sum\limits_{n = 1}^\infty  {\frac{{{2^n}}}{n}{{\left( {4x - 8} \right)}^n}} 
    \)
    \vspace{\stretch{1}}
    \question \(\displaystyle
   \sum\limits_{n = 1}^\infty n\textbf{!}\left(2x+1\right)^{n}
    \)
    \vspace{\stretch{1}}
    \question \(\displaystyle
   \sum\limits_{n = 1}^\infty  {\frac{{{{\left( {x - 6} \right)}^n}}}{{{n^n}}}}
    \)
    \vspace{\stretch{1}}
\end{questions}
\newpage
\large \textbf{Solutions and explanations:} \begin{questions}
    \question \(\displaystyle
    \sum\limits_{n = 1}^\infty  {\frac{{{{\left( { - 1} \right)}^n}n}}{{{4^n}}}{{\left( {x + 3} \right)}^n}}
    \)
    \begin{align*}
    L & = \mathop {\lim }\limits_{n \to \infty } \left| {\frac{{{{\left( { - 1} \right)}^{n + 1}}\left( {n + 1} \right){{\left( {x + 3} \right)}^{n + 1}}}}{{{4^{n + 1}}}}\,\frac{{{4^n}}}{{{{\left( { - 1} \right)}^n}\left( n \right){{\left( {x + 3} \right)}^n}}}} \right|\\ &  = \mathop {\lim }\limits_{n \to \infty } \left| {\frac{{ - \left( {n + 1} \right)\left( {x + 3} \right)}}{{4n}}} \right| \\
    & = \mathop{
    \left|x+3\right|\lim\limits_{n \to \infty}\left(\frac{n+1}{4n}\right)} \\ 
    & = \mathop{
    \frac{1}{4}\left|x+3\right| < 1 \implies \left|x+3\right| < 4 \therefore \text{R.O.C} = 4}
    \end{align*}
    Pretty standard example all around here. Note the fact that we moved over the $\frac{1}{4}$ in order to get the radius of convergence inequality
    
    \question \(\displaystyle
   \sum\limits_{n = 1}^\infty  {\frac{{{2^n}}}{n}{{\left( {4x - 8} \right)}^n}} 
    \)
    \begin{align*}
    L & = \mathop {\lim }\limits_{n \to \infty } \left| {\frac{{{2^{n + 1}}{{\left( {4x - 8} \right)}^{n + 1}}}}{{n + 1}}\,\,\frac{n}{{{2^n}{{\left( {4x - 8} \right)}^n}}}} \right|\\ &  = \mathop {\lim }\limits_{n \to \infty } \left| {\frac{{2n\left( {4x - 8} \right)}}{{n + 1}}} \right|\\ &  = \left| {4x - 8} \right|\mathop {\lim }\limits_{n \to \infty } \frac{{2n}}{{n + 1}}\\ &  = 2\left| {4x - 8} \right| < 1
    \end{align*}
    At this point we want to factor out a 4 from the absolute value term so we can get a coefficient of 1 on the $x$
    \begin{align*}
        L & = 8\left|x-2\right| < 1 \\
        & = \left|x-2\right| < \frac{1}{8} \therefore \text{R.O.C} = \frac{1}{8}
    \end{align*}
    The key takeaway from this problem is the fact that we might often need to factor out values from inside our $x$-term in order to get the coefficient to be 1.

    \newpage

    \large \textbf{Solution and explanation (cont.):} \\ 
    
    \question \(\displaystyle
   \sum\limits_{n = 1}^\infty n\textbf{!}\left(2x+1\right)^{n}
    \)
    \begin{align*}
    L & = \mathop {\lim }\limits_{n \to \infty } \left| {\frac{{\left( {n + 1} \right)!{{\left( {2x + 1} \right)}^{n + 1}}}}{{n!{{\left( {2x + 1} \right)}^n}}}} \right|\\ &  = \mathop {\lim }\limits_{n \to \infty } \left| {\frac{{\left( {n + 1} \right)n!\,\,\left( {2x + 1} \right)}}{{n!}}} \right|\\ &  = \left| {2x + 1} \right|\mathop {\lim }\limits_{n \to \infty } \left( {n + 1} \right) < 1
    \end{align*}
    Now at this point we can clearly see that the limit will be infinite for \textit{almost} all values of $x$. However, there is one such value for which the limit will be finite, $x=-\frac{1}{2}$ since that will cause our $x$-term to go to zero and thus result in the entire term being zero and therefore finite and less than 1. Now because our series will only converge for one value of $x$ we can state that our R.O.C = 0 (see the first special case).
 
    \question \(\displaystyle
   \sum\limits_{n = 1}^\infty  {\frac{{{{\left( {x - 6} \right)}^n}}}{{{n^n}}}}
    \)
    \begin{align*}
        L & = \mathop {\lim }\limits_{n \to \infty } {\left| {\frac{{{{\left( {x - 6} \right)}^n}}}{{{n^n}}}} \right|^{\frac{1}{n}}}\\ &  = \mathop {\lim }\limits_{n \to \infty } \left| {\frac{{x - 6}}{n}} \right|\\ &  = \left| {x - 6} \right|\mathop {\lim }\limits_{n \to \infty } \frac{1}{n} < 1\\ &  = 0 < 1
    \end{align*}
    Now because the limit evaluated to be a constant, 0 which is always less than 1 we know that our power series will converge for all values of $x$ $\therefore$ R.O.C = $\infty$
 
\end{questions}

\newpage \normalsize

\begin{tcolorbox}[title= INTERVAL OF CONVERGENCE,colframe=black,sharp corners,colback=white,colbacktitle=white,coltitle=black]
    \large \textbf{Definition} \\
    \normalsize The interval of convergence is the set of all $x$ for which our power series will converge. Essentially, it is every $x$ that satisfies the inequality for our radius of convergence, 
    \[
    \left|x-c\right| < R
    \]
    \large \textbf{Finding the I.O.C} \\
    \normalsize To find the I.O.C lets first start with the inequality we found in the previous section:
    \[
    \lim\limits_{n\to\infty}\, \left|\frac{a_{n+1}\left({x-c}\right)^{n+1}}{a_n\left({x-c}\right)^{n}} \right| < 1
    \]
    Now just as we did previously lets pull the $\left(x-c\right)$ out of the limit, to give us:
    \[
    \left|x-c\right| \lim\limits_{n\to\infty}\left|\frac{a_{n+1}}{a_n}\right| < 1
    \] 
    Next lets solve for $\left|x-c\right|$
    \[
     \left|x-c\right| < \frac{1}{ \lim\limits_{n\to\infty}\left|\frac{a_{n+1}}{a_n}\right|}
    \]
    Then we can split up the inequality to remove the absolute value bars:
    \[
    \frac{-1}{\lim\limits_{n\to\infty}\left|\frac{a_{n+1}}{a_n}\right|} < \left(x-c\right) < \frac{1}{ \lim\limits_{n\to\infty}\left|\frac{a_{n+1}}{a_n}\right|}
    \]
    And finally we can solve for $x$:
    \[
    \left(\frac{-1}{\lim\limits_{n\to\infty}\left|\frac{a_{n+1}}{a_n}\right|}\right) + c < x < \left(\frac{1}{ \lim\limits_{n\to\infty}\left|\frac{a_{n+1}}{a_n}\right|}\right) + c
    \]
    To make our lives a little easier lets change the notation a bit:
    \[
    \text{Let } S = \left(\frac{-1}{\lim\limits_{n\to\infty}\left|\frac{a_{n+1}}{a_n}\right|}\right) + c
    \]
    \[
    \text{Let } T = \left(\frac{1}{\lim\limits_{n\to\infty}\left|\frac{a_{n+1}}{a_n}\right|}\right) + c
    \]
    Now to determine our final answer we must test for convergence at our endpoints. We can do this by plugging in our endpoints into our power series:
    \[
    \sum\limits_{n=k}^\infty a_n(S-c)^n
    \text{ and }
    \sum\limits_{n=k}^\infty a_n(T-c)^n
    \]
    Convergence for these can then be determined using any of our convergence tests.
    If the series converges at an endpoint then you include it within your interval of convergence. 
\end{tcolorbox}
\newpage

\begin{tcolorbox}[title= INTERVAL OF CONVERGENCE,colframe=black,sharp corners,colback=white,colbacktitle=white,coltitle=black]
    \large \textbf{Special Cases} \normalsize \\
    \begin{enumerate}
        \item If our R.O.C is 0, then our I.O.C is $x=c$
        \item If our R.O.C is infinity, then our I.O.C is $-\infty < x < \infty$ (or in interval notation: \(\left(-\infty, \infty\right)\))
    \end{enumerate}
\end{tcolorbox}
Lets determine the interval of convergence of the previous series
\begin{questions}
    \question \(\displaystyle
    \sum\limits_{n = 1}^\infty  {\frac{{{{\left( { - 1} \right)}^n}n}}{{{4^n}}}{{\left( {x + 3} \right)}^n}}
    \)
    \vspace{\stretch{1}}
    \question \(\displaystyle
   \sum\limits_{n = 1}^\infty  {\frac{{{2^n}}}{n}{{\left( {4x - 8} \right)}^n}} 
    \)
    \vspace{\stretch{1}}
    \question \(\displaystyle
   \sum\limits_{n = 1}^\infty n\textbf{!}\left(2x+1\right)^{n}
    \)
    \vspace{\stretch{1}}
    \question \(\displaystyle
   \sum\limits_{n = 1}^\infty  {\frac{{{{\left( {x - 6} \right)}^n}}}{{{n^n}}}}
    \)
    \vspace{\stretch{1}}
\end{questions}

\newpage

\large \textbf{Solutions and explanations:} \\
\begin{questions}
    \question \(\displaystyle
    \sum\limits_{n = 1}^\infty  {\frac{{{{\left( { - 1} \right)}^n}n}}{{{4^n}}}{{\left( {x + 3} \right)}^n}}
    \) \\
    When we had determined the radius of convergence we setup the following inequality:
    \[
    \left|x+3\right| < 4
    \]
    We can start to find the interval of convergence by expanding this inequality and solving for $x$:
    \begin{align*}
         \left|x+3\right| < 4 
         \implies -4 < x + 3 < 4
         \implies -7 < x < 1
    \end{align*}
    Now at this point we can test for convergence at our end points:
    \begin{align*}
        \sum\limits_{n=1}^\infty \frac{\left(-1\right)^{n}n}{4^n}\left({-7+3}\right)^n 
        & = \sum\limits_{n=1}^\infty \frac{\left(-1\right)^{n}n}{4^n}\left(-4\right)^n \\
        & = \sum\limits_{n=1}^\infty \frac{\left(-1\right)^{n}n}{4^n}\left(-1\right)^n\left(4\right)^n \\
        & = \sum\limits_{n=1}^\infty \left(-1\right)^{2n}n \\
        & = \sum\limits_{n=1}^\infty n
    \end{align*}
    At this point we can use a quick divergence test to see that the power series is going to diverge at $x=-7$. Now for $x=1$: 
    \begin{align*}
        \sum\limits_{n=1}^\infty \frac{\left(-1\right)^{n}n}{4^n}\left(1+3\right)^n
        & = \sum\limits_{n=1}^\infty \frac{\left(-1\right)^{n}n}{4^n}\left(4\right)^n \\
        & = \sum\limits_{n=1}^\infty \left(-1\right)^{n}n
    \end{align*}
    Again a quick divergence test shows that \(\mathop{\lim}\limits_{n\to\infty} \left(-1\right)^{n}n\) does not exist and therefore the power series also diverges at $x=1$ $\therefore$ I.O.C = $(-7, 1) = -7 < x < 1$
    
    \newpage
    \large \textbf{Solutions and explanations (cont.):} 
    
    \question \(\displaystyle
   \sum\limits_{n = 1}^\infty  {\frac{{{2^n}}}{n}{{\left( {4x - 8} \right)}^n}} 
    \)
    When we had determined the radius of convergence we setup the following inequality:
    \[
    \left|x-2\right| < \frac{1}{8}
    \]
    We can start to find the interval of convergence by expanding this inequality and solving for $x$:
    \begin{align*}
         \left|x-2\right| < \frac{1}{8} 
         \implies -\frac{1}{8} < x - 2 < \frac{1}{8}
         \implies \frac{15}{8} < x < \frac{17}{8}
    \end{align*}
    Now at this point we can test for convergence at our end points:

    \begin{align*}
    \sum\limits_{n = 1}^\infty  {\frac{{{2^n}}}{n}{{\left( {\frac{{15}}{2} - 8} \right)}^n}} & = \sum\limits_{n = 1}^\infty  {\frac{{{2^n}}}{n}{{\left( { - \frac{1}{2}} \right)}^n}} \\ &  = \sum\limits_{n = 1}^\infty  {\frac{{{2^n}}}{n}\frac{{{{\left( { - 1} \right)}^n}}}{{{2^n}}}} \\ &  = \sum\limits_{n = 1}^\infty  {\frac{{{{\left( { - 1} \right)}^n}}}{n}}
    \end{align*}
    We know this series to be the alternating harmonic series so the power series converges at $x=\frac{15}{8}$
    \begin{align*}\sum\limits_{n = 1}^\infty  {\frac{{{2^n}}}{n}{{\left( {\frac{{17}}{2} - 8} \right)}^n}} & = \sum\limits_{n = 1}^\infty  {\frac{{{2^n}}}{n}{{\left( {\frac{1}{2}} \right)}^n}} \\ &  = \sum\limits_{n = 1}^\infty  {\frac{{{2^n}}}{n}\frac{1}{{{2^n}}}} \\ &  = \sum\limits_{n = 1}^\infty  {\frac{1}{n}} \end{align*}
    This is the alternating harmonic series so the power series diverges at $x=\frac{17}{8} \therefore$ I.O.C = $\left[\frac{15}{8}, \frac{17}{8}\right) = \frac{15}{8} \leq x < \frac{17}{8}$
    \newpage
    \question \(\displaystyle
   \sum\limits_{n = 1}^\infty n\textbf{!}\left(2x+1\right)^{n}
    \) \\
    Because R.O.C = 0 we know that the power series is only going to converge at $x=c \therefore$ I.O.C is $x = -\frac{1}{2}$ 
    \question \(\displaystyle
   \sum\limits_{n = 1}^\infty  {\frac{{{{\left( {x - 6} \right)}^n}}}{{{n^n}}}}
    \) \\
    Because our R.O.C $= \infty$ we know that our I.O.C $= \left(-\infty, \infty\right) = -\infty < x < \infty$
\end{questions}
\vspace{\stretch{1}}
\begin{tcolorbox}[title=NOTATION,colframe=black,sharp corners,colback=white,colbacktitle=white,coltitle=black]
    Here is some common notation that I used and what it all means:
    \begin{enumerate}
        \item '$\therefore$' = 'therefore'
        \item '$\implies$' = 'implies' or 'therefore' (usually used more in computational contexts)
        \item Interval Notation: A way to notate a set of values of $x$. Parenthesis are exclusive and brackets are inclusive. Can be used interchangable with inequalities. 
    \end{enumerate}
\end{tcolorbox}
\end{document}