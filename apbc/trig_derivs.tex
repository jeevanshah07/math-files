\documentclass[addpoints]{exam}
\usepackage{amsmath}
\usepackage{amsfonts}
\usepackage{amssymb}
\usepackage[most]{tcolorbox}
\usepackage{tikz}
\usepackage{pgfplots}
\usepackage{mdframed}
\usepackage{hyperref}
\usepackage{amsthm}
\usepackage[x11names, svgnames]{xcolor}
\usepackage{cancel}
\usepackage{tocbibind}

\usetikzlibrary{decorations.markings}

\marksnotpoints
\pointsinrightmargin
\bracketedpoints
\printanswers

\hypersetup{
  colorlinks=true,
  linkcolor=blue,
  linktoc=subsection,
  filecolor=magenta,
  urlcolor=blue,
  pdfpagemode=FullScreen,
}

\theoremstyle{definition}
\newtheorem*{definition}{Definition}

\theoremstyle{plain}
\renewcommand\qedsymbol{$\blacksquare$}
\newtheorem{theorem}{Theorem}

\renewcommand{\implies}{\Rightarrow}

\urlstyle{same}

\pagestyle{headandfoot}
\firstpageheadrule
\runningheadrule
\firstpageheader{Calc Prep}{Intro to Calculus}{Shah}
\runningheader{Intro to Calculus}{}{Shah}
\firstpagefooter{}{}{}
\runningfooter{ }{\thepage}{ }

\begin{document}
\section{Notes}
Last time we learned about derivatives of polynomials with the quotient and power rule. Today we're going to begin taking on more complex functions! 
\begin{tcolorbox}[breakable, title=\subsection{DERIVATIVES OF TRIG FUNCTIONS}, colframe=black, sharp corners, colback=Azure4!70, colbacktitle=DodgerBlue3!60, coltitle=black]
    \begin{enumerate}
        \begin{minipage}{0.45\linewidth}
            \item $\displaystyle \frac{d}{dx}\left[\sin\left(x\right)\right] = \cos\left(x\right)$   
            \item $\displaystyle \frac{d}{dx}\left[\cos\left(x\right)\right] = -\sin\left(x\right)$   
            \item $\displaystyle \frac{d}{dx}\left[\tan\left(x\right)\right] = \sec^2\left(x\right)$   
        \end{minipage}
        \hfill 
        \begin{minipage}{0.45\linewidth}
            \item $\displaystyle \frac{d}{dx}\left[\csc\left(x\right)\right] = -\csc\left(x\right)\cot\left(x\right)$   
            \item $\displaystyle \frac{d}{dx}\left[\sec\left(x\right)\right] = \sec\left(x\right)\tan\left(x\right)$   
            \item $\displaystyle \frac{d}{dx}\left[\cot\left(x\right)\right] = -\csc^{2}\left(x\right)$   
        \end{minipage}
    \end{enumerate}
\end{tcolorbox}
Find $f'$ for the following functions
\begin{questions}
    \question $\displaystyle\,f(x)=2\sin\left(x\right)$
    \begin{solution}[\stretch{0.5}]
        \[ \frac{d}{dx}\left[2\sin\left(x\right)\right] = 2\cos\left(x\right) \]
    \end{solution}
    
    \question $\displaystyle\,f(x)=\frac{1}{\tan\left(x\right)}$
    \begin{solution}[\stretch{0.5}]
        \[ \frac{1}{\tan\left(x\right)} = \cot\left(x\right) \Rightarrow \frac{d}{dx}\left[\cot\left(x\right)\right] = -\csc\left(x\right)\cot\left(x\right) \]
    \end{solution}
    
    \question Evaluate the following expressions 
    \vspace{0.1in}\newline
    \[ \frac{d}{dx}\left(\frac{d}{dx}\left(\frac{d}{dx}\left(\frac{d}{dx}\left[\sin\left(x\right)\right]\right)\right)\right) \]
    \begin{solution}[\stretch{1}]
        \vspace{0.1in}\newline
        \begin{align*}
            \frac{d}{dx}\left(\frac{d}{dx}\left(\frac{d}{dx}\left(\frac{d}{dx}\left[\sin\left(x\right)\right]\right)\right)\right) &= 
            \frac{d}{dx}\left(\frac{d}{dx}\left(\frac{d}{dx}\left[\cos\left(x\right)\right]\right)\right) \\
            &= \frac{d}{dx}\left(\frac{d}{dx}\left[-\sin\left(x\right)\right]\right) \\
            &= \frac{d}{dx}\left[-\cos\left(x\right)\right] \\
            &= \boxed{\sin\left(x\right)} \\
        \end{align*}
    \end{solution}
\end{questions}

\newpage 

\begin{tcolorbox}[breakable, title=\subsection{CHAIN RULE}, colframe=black, sharp corners, colback=Azure4!30, colbacktitle=Firebrick2!60, coltitle=black]
    A large majority of the functions encountered in your math career are compositions of functions, something in the form $h(x) = f(g(x))$ where $f$ and $g$ are differentiable functions. We can find $h'$ as following...
    \[
        h(x) = f(g(x)) \Rightarrow f'(g(x))g'(x)
    \]
    An alternate form you may encounter is the following:
    \[
        \frac{dy}{dx} = \frac{dy}{du} \cdot \frac{du}{dx}
    \]
    if $y=f(u)$ and $u=g(x)$ where $f$ and $g$ are differentiable functions. 
\end{tcolorbox}
Compute the following derivatives using the \textbf{chain rule}
\begin{questions}
    \question $\displaystyle\,\frac{\mathrm{d}}{\mathrm{d}x}\left[\left(2x+1\right)^2\right]$
    \begin{solution}[\stretch{0.5}]
        \[ = 2\left(x+1\right)\cdot\,2 = 4\left(x+1\right)\]
    \end{solution}
    
    \question $\displaystyle\,\frac{\mathrm{d}}{\mathrm{d}x}\left[\sin\left(3x^2\right)\right]$
    \begin{solution}[\stretch{.5}]
        \[ = \cos\left(3x^2\right) \cdot 6x = 6x\cos\left(3x^2\right) \]
    \end{solution}
    
    \question $\displaystyle\,\frac{\mathrm{d}}{\mathrm{d}x}\left[\sqrt{\sec\left(2x\right)}\right]$
    \begin{solution}[\stretch{.5}]
        \[ = \frac{1}{2}\sec^{-1/2}\left(2x\right) \cdot \sec\left(2x\right)\tan\left(2x\right) \cdot 2 = \sec^{-1/2}\left(2x\right)\sec\left(2x\right)\tan\left(2x\right) = \sec^{1/2}\left(x\right)\tan\left(x\right)\]
    \end{solution}
\end{questions}

\newpage 

\begin{tcolorbox}[breakable, title=\subsection{IMPLICIT DIFFERENTIATION}, colframe=black, sharp corners, colback=Azure4!70, colbacktitle=DeepPink2!60, coltitle=black]
    Sometimes you may encounter \emph{relations} that aren't functions - to find the derivative of these relations we need to use \textbf{implicit differentiation}. 
    \begin{tcolorbox}[breakable, title=\subsection{STEPS FOR IMPLICIT DIFFERENTIATION}, colframe=black, sharp corners, colback=Azure4!30, colbacktitle=DeepPink1!30, coltitle=black]
        \begin{enumerate}
            \item Derive each side of the equation following all derivative rules \underline{\textbf{however} each $y$ gets multiplied by a $y'$}
            \item Collect all the terms with $y'$ on one side of the equation
            \item Factor out $y'$
            \item Solve for $y'$
        \end{enumerate}
    \end{tcolorbox}
\end{tcolorbox}
Find the derivative of the following expressions using \emph{implicit differentiation}
\begin{questions}
    \question $x^2+y^2 = 0$
    \begin{solution}[\stretch{1}]
        \begin{align*}
            \frac{d}{dx}\left[x^2+y^2 = 25\right] &\Rightarrow 2x + 2yy' = 0 \\
            &\Rightarrow 2x = -2yy' \\
            &\Rightarrow \frac{2x}{2y} = y' \\
            &\Rightarrow \boxed{\frac{x}{y} = y' = \frac{dy}{dx}}
        \end{align*}
    \end{solution}

    \question $x^2\tan\left(y\right) + y^10\sec\left(x\right) = 2$
    \begin{solution}[\stretch{1}]
        \begin{align*}
            \frac{d}{dx}\left[x^2\tan\left(y\right) + y^{10}\sec\left(x\right) = 2\right] &\Rightarrow 2x\tan\left(y\right) + x^2\sec^{2}\left(y\right)y' + 10y^{9}y'\sec\left(x\right) + y^{10}\sec\left(x\right)\tan\left(x\right) = 0 \\ 
            &\Rightarrow 2x\tan\left(y\right) + y^{10}\sec\left(x\right)\tan\left(x\right) = -x^{2}y'\sec^{2}\left(y\right) - 10y^{9}y'\sec\left(x\right) \\
            &\Rightarrow 2x\tan\left(y\right) + y^{10}\sec\left(x\right)\tan\left(x\right) = y'\left(-x^{2}\sec^{2}\left(y\right) - 10y^{9}\sec\left(x\right)\right) \\
            &\Rightarrow \boxed{\frac{2x\tan\left(y\right) + y^{10}\sec\left(x\right)\tan\left(x\right)}{-x^2\sec^{2}\left(y\right) - 10y^{9}\sec\left(x\right)} = y'}
        \end{align*}
    \end{solution}
\end{questions}

\newpage 

\begin{tcolorbox}[breakable, title=\subsection{APPLICATION: PHYSICS}, colframe=black, sharp corners, colback=Azure4!30, colbacktitle=DarkOrchid2!60, coltitle=black]
    Physics, as we know, is based off of math and specifically calculus. A large of physics is about predicting the motion of objects, something commonly done by analyzing an object's positions, velocity, and acceleration. We will notice the following relationship between these three quantities: $a(t) = v'(t) = s''(t)$ where $a(t)$ is the acceleration function, $v(t)$ is the velocity function, and $s(t)$ is the position function\footnote{in case you're wondering, the $s$ stands for 'space'}.
\end{tcolorbox}
\begin{questions}
    \question An object's position function is given by \[ s(t) = \sin\left(6t\right) - \left(32t + 2\right)^2\]
    \begin{parts}
        \part Find the velocity function, $v(t)$
        \begin{solution}
            \begin{align*}
                v(t) = s'(t) = 6\cos\left(6t\right) - 64\left(32t + 2\right)
            \end{align*}
        \end{solution}
        \part Find the acceleration function, $a(t)$ 
        \begin{solution}
            \[
            a(t) = v'(t) = s''(t) = -36\sin\left(6t\right) - 2048
            \]
        \end{solution}
    \end{parts}
\end{questions}

\newpage 

\section{Problem Set}
Complete the following problems. Answers are in the next section. For help consider using \href{https://derivative-calculator.net}{\underline{derivative calculator}}, SchoolAI, or asking me, \underline{before checking the answer key}! {\texttt{[Calculator allowed for 3, 4, 5]}}

\begin{questions}
    \question Find the derivatives of the following functions
    \begin{parts}
        \part $\displaystyle \frac{d}{dx} \left( (3x^2 + 5x) \sin(x^3 + 2) \right)$

        \part $\displaystyle \frac{d}{dx} \left( \frac{\cos(2x^2 + 1)}{x^4} \right)$

        \part $\displaystyle \frac{d}{dx} \left( \tan(x^2 + 1) \right)$
        
        \part $\displaystyle \frac{d}{dx} \left( \csc(x^2 + 3x + 2)\right)$
        
        \part $\displaystyle \frac{d}{dx} \left( \sin\left(\sec\left(x\right)\right) \right)$
        
        \part $\displaystyle \frac{d}{dx} \left( \frac{\cos(5x^4 + 2x)}{2x} \right)$
        
        \part $\displaystyle \frac{d}{dx} \left( 4x^5 \cot(3x^2 - 1) \right)$
        
        \part $\displaystyle \frac{d}{dx} \left(  \sin\left(\left(x^2 + 1\right)^3\right) \right)$
        
        \part $\displaystyle \frac{d}{dx} \left( x^2 \sec(x^3 + 5x) \right)$
        
        \part $\displaystyle \frac{d}{dx} \left( (3x^4 - 2x) \cot(x^6 + 3x) \right)$        
    \end{parts}

    \question Use implicit differentiation to find the derivative of the following
    \begin{parts}

        \part $\displaystyle \frac{d}{dx} \left( x^3 + y^3 = 6xy \right)$
        
        \part $\displaystyle \frac{d}{dx} \left( x^2y + y^2 = 4 \right)$
        
        \part $\displaystyle \frac{d}{dx} \left( x^2 + y^2 + z^2 = 1 \right)$
        
        \part $\displaystyle \frac{d}{dx} \left( x^3 + y^3 = 3x^2y + 3xy^2 \right)$
        
        \part $\displaystyle \frac{d}{dx} \left( x^2 + y^2 + xy = 10 \right)$
        
        \part $\displaystyle \frac{d}{dx} \left( x^4 + y^4 = 4x^2y^2 \right)$
        
        \part $\displaystyle \frac{d}{dx} \left( \sin(xy) = x + y \right)$
        
    \end{parts}
\end{questions}

\newpage

\section{Answers}
\begin{questions}
    \question \begin{parts}
        \part $\displaystyle \frac{d}{dx} \left( (3x^2 + 5x) \sin(x^3 + 2) \right) = (6x + 5) \sin(x^3 + 2) + (3x^2 + 5x) \cos(x^3 + 2) \cdot 3x^2$
        
        \part $\displaystyle \frac{d}{dx} \left( \frac{\cos(2x^2 + 1)}{x^4} \right) = \frac{-8x^3 \sin(2x^2 + 1) - 4x^3 \cos(2x^2 + 1)}{x^8}$
        
        \part $\displaystyle \frac{d}{dx} \left( \tan(x^2 + 1) \right) = 2x \sec^2(x^2 + 1)$
        
        \part $\displaystyle \frac{d}{dx} \left( \csc(x^2 + 3x + 2) \right) = -\cot(x^2 + 3x + 2) \cdot (2x + 3) \csc(x^2 + 3x + 2)$
        
        \part $\displaystyle \frac{d}{dx} \left( \sin\left(\sec\left(x\right)\right) \right) = \cos(\sec(x)) \cdot \sec(x) \cdot \tan(x)$
        
        \part $\displaystyle \frac{d}{dx} \left( \frac{\cos(5x^4 + 2x)}{2x} \right) = \frac{-20x^3 \sin(5x^4 + 2x) \cdot 2x - \cos(5x^4 + 2x) \cdot 2}{4x^2}$
        
        \part $\displaystyle \frac{d}{dx} \left( 4x^5 \cot(3x^2 - 1) \right) = 20x^4 \cot(3x^2 - 1) - 24x^5 \csc^2(3x^2 - 1) \cdot 6x$
        
        \part $\displaystyle \frac{d}{dx} \left( \sin\left(\left(x^2 + 1\right)^3\right) \right) = 6x(x^2 + 1)^2 \cos\left(\left(x^2 + 1\right)^3\right)$
        
        \part $\displaystyle \frac{d}{dx} \left( x^2 \sec(x^3 + 5x) \right) = 2x \sec(x^3 + 5x) + x^2 \sec(x^3 + 5x) \cdot \tan(x^3 + 5x) \cdot (3x^2 + 5)$
        
        \part $\displaystyle \frac{d}{dx} \left( (3x^4 - 2x) \cot(x^6 + 3x) \right) = (12x^3 - 2) \cot(x^6 + 3x) - (3x^4 - 2x) \csc^2(x^6 + 3x) \cdot (6x^5 + 3)$    
    \end{parts}   

    \question \begin{parts}
        \part $\displaystyle \frac{dy}{dx} = \frac{6y - 3x^2}{3y^2 - 6x}$
        
        \part $\displaystyle \frac{dy}{dx} = \frac{-2xy}{x^2 + 2y}$
        
        \part $\displaystyle x + y \frac{dy}{dx} + z \frac{dz}{dx} = 0$
        
        \part $\displaystyle \frac{dy}{dx} = \frac{6xy - 3x^2 - 3y^2}{3y^2 - 3x^2 - 6xy}$
        
        \part $\displaystyle \frac{dy}{dx} = \frac{-2x - y}{2y + x}$
        
        \part $\displaystyle \frac{dy}{dx} = \frac{8xy^2 - 4x^3}{4y^3 - 8x^2y}$
        
        \part $\displaystyle \frac{dy}{dx} = \frac{1 - y \cos(xy)}{x \cos(xy) - 1}$
    \end{parts}
\end{questions}
\end{document}