\documentclass{article}
\usepackage{amsmath}
\usepackage{amssymb}
\usepackage{amsthm}
\usepackage{fancyhdr}

\renewcommand\qedsymbol{$\blacksquare$}

\newtheorem{theorem}{Theorem}

\pagestyle{fancy}
\fancyhead{}
\fancyhead[L]{Jeevan Shah}
\fancyhead[R]{\today}

\begin{document}
\begin{theorem}[Divisibility by a Prime]
Any integer $n>1$ is divisible by a prime number.
\end{theorem}

\begin{theorem}
For all integers $n$, if $n>2$ then there is a prime number $p$ such that $n<p<n\textbf{!}$
\end{theorem}

\begin{proof}
  Assume that $n$ is an integer greater than 2 and that $p$ is a prime number such that $p \mid \left(n\textbf{!} - 1\right)$ \textit{[by the prime divisibility theorem]} and that $p\geq\,n$. Then, by definition of factorial, $p \mid n\textbf{!}$. So, 

  \begin{equation}
    n\textbf{!} \equiv 0 \pmod{p}
  \label{eq:1}
  \end{equation}

  However, by assumption since $p \mid \left(n\textbf{!}-1\right)$ we can say,

  \begin{equation}
    n\textbf{!} - 1 \equiv 0 \pmod{p}
  \label{eq:2}
  \end{equation}

  Manipulating the left hand side of (\ref{eq:1}) to match (\ref{eq:2}) gives,

  \begin{align}
    n\textbf{!} &\equiv 0 \pmod{p} \nonumber \\ 
    n\textbf{!} - 1 &\equiv 0 - 1 \pmod{p} \nonumber \\ 
    n\textbf{!} - 1 &\equiv -1 \pmod{p} \nonumber \\ 
    \implies 0 &\equiv -1 \pmod{p} \label{eq:3}
  \end{align}

  But, in order for equation~\ref{eq:3} to be true $p$ must be equal to $1$ \textit{[for the sake of modular arithmatic we assume our domain to be $\mathbb{Z}^{+}$]}. Thus we have reached a contradiction since $p \geq n$ and $n > 2$ so $p \ne 1$. So, $p > n$ and \\ $p \mid \left(n\textbf{!}-1\right) \implies p \leq \left(n\textbf{!}-1\right)$. Because $\left(n\textbf{!}-1\right) < n\textbf{!}$, we know that \\ $p \leq \left(n\textbf{!}-1\right) < n\textbf{!}$ so, $n < p < n\textbf{!}$ \textit{[which was to be proved.]}
\end{proof}
\end{document}
