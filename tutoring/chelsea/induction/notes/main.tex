\documentclass[addpoints]{exam}
\usepackage{amsmath}
\usepackage{amsfonts}
\usepackage{amssymb}
\usepackage[most]{tcolorbox}
\usepackage{tikz}
\usepackage{pgfplots}
\usepackage{mdframed}
\usepackage{hyperref}
\usepackage{amsthm}
\usepackage{xcolor}

\marksnotpoints
\pointsinrightmargin
\bracketedpoints
\printanswers

\hypersetup{
  colorlinks=true,
  linkcolor=blue,
  filecolor=magenta,
  urlcolor=blue,
  pdfpagemode=FullScreen,
}

\renewcommand\qedsymbol{$\blacksquare$}
\newtheorem{theorem}{Theorem}

\urlstyle{same}

\pagestyle{headandfoot}
\firstpageheadrule
\runningheadrule
\firstpageheader{Pre Calc Prep}{Proof by Induction}{Shah}
\runningheader{Proof by Induction Notes}{}{Shah}
\firstpagefooter{}{}{}
\runningfooter{ }{\thepage}{ }

\begin{document}
\begin{tcolorbox}[breakable, title=INDUCTION INTRO, colframe=black, sharp corners, colback=white, colbacktitle=white, coltitle=black]
	\Large\textbf{Introduction}
	\newline\normalsize So far in your mathematical career, your experience with proofs is only that of two column proofs in geometry. However, you may be surprised to learn that two column proofs are, in fact, not the only type of proofs used in math and are often not used at all. This is where we introduce a new method of proof: \textbf{proof by induction}.
	\begin{tcolorbox}[breakable, title=INDUCTION STEPS, colframe=black, sharp corners, colback=white, colbacktitle=white, coltitle=black]
		Let $P(n)$ be some proposition that is defined for all integers $n$ ($n \in \mathbb{N}$\footnote{Induction is standardly done with the natural numbers, however it can be done with all integers}), and let $a$ be a certain fixed integer.
		\begin{enumerate}
			\item Show that $P(a)$ is true (called the basis step)
			\item Show that for all integers $k \geq a$, if $P(k)$ is true then $P(k+1)$ is also true (called the inductive step)
		\end{enumerate}
		Completing these steps shows the truthfullness of the statement "for all integers $n \geq a, P(n)$"
	\end{tcolorbox}
\end{tcolorbox}
\vspace{0.2in}
Induction is often best learnt through an example. Let's take a look at the following theorem and see how we can prove it using induction.
\begin{theorem}[Sum of the first $n$ integers]
	For all integers $n \geq 1$
	\[
		1 + 2 + 3 + \cdots + n = \frac{n\left(n+1\right)}{2}
	\]
\end{theorem}
\ifprintanswers
	\begin{proof}
		Let the property $P(n)$ be the formula
		\[
			1 + 2 + 3 + \cdots + n = \frac{n(n+1)}{2}
		\]
		\newline\textbf{Show that $P(1)$ is true:} To establish $P(1)$ we must show that
		\[
			1 = \frac{1(1+1)}{2}
		\]
		Since the left hand side of our equation is $1$ and the right hand side of the equation is $\displaystyle\,\frac{1(1+1)}{2}=\frac{2}{2} = 1$ and now that our LHS and RHS equal the same quantity we can conclude that they are equivalent.
		\newline\textbf{Show that if $P(k)$ is true, so is $P(k+1)$}: First lets assume that $P(k)$ is true and so
		\[
			1 + 2 + 3 + \cdots + k = \frac{k(k+1)}{2}
		\]
		Now we must show that $P(k+1)$ is also true or that
		\[
			1 + 2 + 3 + \cdots + (k+1) = \frac{\left(k+1\right)\left[\left(k+1\right)+1\right]}{2} = \frac{\left(k+1\right)\left(k+2\right)}{2}
		\]
		Just like in our basis step, we will form this equality by showing that the LHS and the RHS are equivalent. We will start with the LHS and note that we can rewrite it in the following way,
		\begin{align*}
			1 + 2 + 3 + \cdots + (k+1) & = \textcolor{blue}{1 + 2 + 3 + \cdots + k} + (k + 1)                        \\
			                           & = \frac{k(k+1)}{2} + (k+1) \tag{by making the second to last term explicit} \\
			                           & = \frac{k(k+1)}{2} + \frac{2(k+1)}{2}                                       \\
			                           & = \frac{(k+1)(k+2)}{2} \tag{by algebra and factoring}
		\end{align*}
		Now our LHS and RHS are equivalent to the same quantity and are equivalent. So, by using proof by induction our theorem holds true.
	\end{proof}
\else
	\begin{proof}
		Let the property $P(n)$ be
		\vspace{0.5in}
		\newline\textbf{Show that $P(1)$ is true:} To establish $P(1)$ we must show that
		\[
			1 = \phantom{\frac{1(2)}{2}}
		\]
		Since the left hand side of our equation is $1$ and the right hand side of the equation is $\displaystyle\,\frac{1(1+1)}{2}=\frac{2}{2} = 1$ and now that our LHS and RHS equal the same quantity we can conclude that they are equivalent.
		\newline\textbf{Show that if $P(k)$ is true, so is $P(k+1)$}: First lets assume that $P(k)$ is true and so
		\[
			\phantom{1 + 2 + 3 + \cdots + k} = \frac{k(k+1)}{2}
		\]
		Now we must show that $P(k+1)$ is also true or that
		\[
			1 + 2 + 3 + \cdots + (k+1) = \frac{\left(k+1\right)\left[\left(k+1\right)+1\right]}{2} = \frac{\left(k+1\right)\left(k+2\right)}{2}
		\]
		Just like in our basis step, we will form this equality by showing that the LHS and the RHS are equivalent. We will start with the LHS and note that we can rewrite it in the following way,
		\begin{align*}
			1 + 2 + 3 + \cdots + (k+1) & = \textcolor{blue}{1 + 2 + 3 + \cdots + k} + (k + 1)                        \\
			                           & = \frac{k(k+1)}{2} + (k+1) \tag{by making the second to last term explicit} \\
			                           & = \frac{k(k+1)}{2} + \phantom{\frac{2(k+1)}{2}}                             \\
			                           & = \frac{(k+1)(k+2)}{2} \tag{by algebra and factoring}
		\end{align*}
		Now our LHS and RHS are equivalent to the same quantity and are equivalent. So, by using proof by induction our theorem holds true.
	\end{proof}
\fi
\noindent\makebox[\linewidth]{\hrulefill}
\vspace{0.1in}
\newline A few notes: First you'll notice that similar to proving trig identities we don't want to treat the equals sign like an equation - instead we want to manipulate both sides independently until they are equivalent. Next, you'll notice that we made the second to last term explicit by using our assumption that $P(k)$ holds true, this is an extremely common theme in proof by induction and is something you should become extremely comfortable with.
\noindent\makebox[\linewidth]{\hrulefill}
\begin{theorem}[Sum of a Geometric Series]
	For any real number $r$ not equal to $1$ and any integers $n \geq 0$
	\[
		\sum\limits_{i=1}^{n} r^{i} = \frac{r^{n+1} - 1}{r-1}
	\]
\end{theorem}
\ifprintanswers
	\begin{proof}
		Let the property $P(n)$ be the formula
		\[
			\sum\limits_{i=0}^{n} r^{i} = \frac{r^{n+1}-1}{r-1}
		\]
		\textbf{Show that $P(0)$ is true}: We will do this by showing that the LHS and the RHS of the equality are equivalent.
		\begin{minipage}[t]{0.45\linewidth}
			\begin{center}
				\underline{LHS}
			\end{center}
			\[
				\sum\limits_{i=0}^{0} r^{i} = r^{0} = 1
			\]
		\end{minipage}
		\hfill
		\begin{minipage}[t]{0.45\linewidth}
			\begin{center}
				\underline{RHS}
			\end{center}
			\[
				\frac{r^{0+1}-1}{r-1} = \frac{r^1-1}{r-1} = \frac{r-1}{r-1} = 1
			\]
		\end{minipage}
		\vspace{0.2in}\newline Since our LHS and RHS equal the same quantity they are equivalent and $P(0)$ holds true.
		\newline \textbf{Show that if $P(k)$ is true so is $P(k+1)$:} First lets assume that $P(k)$ is true and so
		\[
			\sum\limits_{i=0}^{k} r^i = \frac{r^{k+1}-1}{r-1}
		\]
		Now, we must show that $P(k+1)$ is also true or that
		\[
			\sum\limits_{i=0}^{k+1} r^i = \frac{r^{(k+1)+1}-1}{r-1} = \frac{r^{k+2}-1}{r-1}
		\]
		To show our equality we will work with the LHS,
		\begin{align*}
			\sum\limits_{i=0}^{k+1} r^i & = r^{k+1} + \sum\limits_{i=0}^{k} r^i \tag{by stripping out the last term}                   \\
			                            & = r^{k+1} + \frac{r^{k+1}-1}{r-1} \tag{by making the summation explicit}                     \\
			                            & = \frac{r^{k+1}\left(r-1\right)}{r-1} + \frac{r^{k+1}-1}{r-1}                                \\
			                            & = \frac{r^{k+2} - r^{k+1} + r^{k+1}-1}{r-1} \tag{using the fact that $r^{k+1}(r) = r^{k+2}$} \\
			                            & = \frac{r^{k+2}-1}{r-1}
		\end{align*}
		Now our LHS and RHS are equivalent to the same quantity and are equivalent thus proving the validity of our theorem.
	\end{proof}
\else
	\begin{proof}
		Let the property $P(n)$ be the formula
		\vspace{0.5in}
		\newline \textbf{Show that $P(0)$ is true}:
		\vspace{1.5in}
		\newline \textbf{Show that if $P(k)$ is true so is $P(k+1)$:}
		\vspace{2in}
		\newline
	\end{proof}
\fi
\newpage
\begin{tcolorbox}[breakable, title=FACTORIAL REVIEW, colframe=black, sharp corners, colback=white, colbacktitle=white, coltitle=black]
	\Large\textbf{Intro and Definition}
	\normalsize
	\begin{tcolorbox}[breakable, title=DEFINITION, colframe=black, sharp corners, colback=white, colbacktitle=white, coltitle=black]
		For all positive integers $n$, $n$ factorial is defined as follows
		\[
			n\textbf{!} = 1 * 2 * 3 * \cdots * n
		\]
		The factorial function is defined for all non negative integers ($\mathbb{Z}^{nonneg}$) with $0\textbf{!} = 1$ and $1\textbf{!} = 1$.
	\end{tcolorbox}
	The real world application of the factorial can be thought of the number of different ways to arrange $n$ objects. This also helps explain why $0\textbf{!} = 1$, as there is only 1 way to arrange no objects.
	\vspace{0.1in}
	\noindent\makebox[\linewidth]{\hrulefill}
	\newline
	\Large\textbf{Properties and Important Notes}
	\normalsize
	\begin{tcolorbox}[breakable, title=PROPERTIES, colframe=black, sharp corners, colback=white, colbacktitle=white, coltitle=black]
		\begin{enumerate}
			\item Factorial can be defined recursively meaing that
			      \[
				      n\textbf{!} = n * \left(n-1\right)\textbf{!}
			      \]
			\item The factorial operation is \textbf{not}a distributive operator
			      \[
				      (n+1)\textbf{!} \ne n\textbf{!} + 1\textbf{!}
			      \]
			\item This also means that
			      \[
				      \left(2n\right)\textbf{!} \ne 2 * n\textbf{!}
			      \]
		\end{enumerate}
	\end{tcolorbox}
\end{tcolorbox}
\vspace{0.2in}
\Large\textbf{Examples}\normalsize
\begin{questions}
	\question Find $n\textbf{!}$ for $n=1$ to $n=5$
	\begin{solution}[\stretch{1}]
		\begin{align*}
			1\textbf{!} & = 1                       \\
			2\textbf{!} & = 2 * 1 = 2               \\
			3\textbf{!} & = 3 * 2 * 1 = 6           \\
			4\textbf{!} & = 4 * 3 * 2 * 1 = 24      \\
			5\textbf{!} & = 5 * 4 * 3 * 2 * 1 = 120
		\end{align*}
	\end{solution}

	\question Evaluate $\displaystyle\,\frac{8\textbf{!}}{5\textbf{!}}$
	\begin{solution}[\stretch{1}]
		\begin{align*}
			\frac{8\textbf{!}}{5\textbf{!}} & = \frac{8 * 7 * 6 * 5\textbf{!}}{5\textbf{!}} \\
			                                & = 8 * 7 * 6 = 336
		\end{align*}
	\end{solution}

	\question Simplify the following expression $\displaystyle\,\frac{\left(n+2\right)\textbf{!}}{n\left(n-1\right)\textbf{!}}$
	\begin{solution}[\stretch{1}]
		\begin{align*}
			\frac{\left(n+2\right)\textbf{!}}{n\left(n-1\right)\textbf{!}} & = \frac{\left(n+2\right)\textbf{!}}{n\textbf{!}}                               \\
			                                                               & = \frac{\left(n+2\right)\left(n+1\right)\left(n\textbf{!}\right)}{n\textbf{!}} \\
			                                                               & = \left(n+2\right)\left(n+1\right)
		\end{align*}
	\end{solution}
\end{questions}
\end{document}
