\documentclass[addpoints]{exam}
\usepackage{amsmath}
\usepackage{amsfonts}
\usepackage[most]{tcolorbox}
\usepackage{tikz}
\usepackage{pgfplots}
\usepackage{mdframed}
\usepackage{hyperref}
\usepackage{amsthm}
\usepackage{amssymb}

\marksnotpoints
\pointsinrightmargin
\bracketedpoints
\printanswers
\renewcommand\qedsymbol{$\blacksquare$}

\hypersetup{
  colorlinks=true,
  linkcolor=blue,
  filecolor=magenta,
  urlcolor=blue,
  pdfpagemode=FullScreen,
}

\urlstyle{same}

\pagestyle{headandfoot}
\firstpageheadrule
\runningheadrule
\firstpageheader{Pre Calc Prep}{Induction PSET}{Shah}
\runningheader{Induction Problem Set}{}{Shah}
\firstpagefooter{}{}{}
\runningfooter{ }{\thepage}{ }

\begin{document}
  \vspace{1in}

  \noindent\makebox[\textwidth]{Name:\enspace\hrulefill}

  \vspace{.1in}

  \begin{center}
      \fbox{\fbox{\parbox{5.5in}{\centering
      Answer the questions in the spaces provided on the following pages.  If you run out of room for an answer, continue on the back of the page. Show \textbf{all} your work!}}}
  \end{center}
  
  \begin{questions}
    \question Prove the following using proof by induction
    \begin{center}
      For all positive integers, $n$, $\displaystyle\,1^2 + 2^2 + 3^2 + \cdots + n^2 = \frac{n(n+1)(2n+1)}{6}$
    \end{center}
    \begin{solution}[\stretch{1}]
      \begin{proof}
        Let $P(n)$ be the formula \[\displaystyle\,1^2 + 2^2 + 3^2 + \cdots + n^2 = \frac{n(n+1)(2n+1)}{6}\]
        \textbf{Show that $P(1)$ is true:} To show that $P(1)$ is true we will show that LHS and the RHS are equivalent. The LHS is simply $1^2=1$ and the RHS is $\displaystyle\, \frac{1(1+1)(2(1)+1)}{6} = \frac{1(2)(3)}{6} = 1$ and so the LHS and RHS equal the same quantity and thus must be equivalent, proving the truthfullness of $P(1)$
        \newline\textbf{Show that if $P(k)$ is true, so is $P(k+1)$:} First, we will assume that $P(k)$ is true or that 
        \[
          1^2 + 2^2 + 3^2 + \cdots + k^2 = \frac{k(k+1)(2k+1)}{6}
        \]
        Next, we will need to show that $P(k+1)$ is true or that 
        \[
          1^2 + 2^2 + 3^2 + \cdots + (k+1)^2 = \frac{(k+1)\left[\left(k+1\right)+1\right]\left[2\left(k+1\right)+1\right]}{6} = \frac{(k+1)(k+2)(2k+3)}{6}
        \]
        We can rewrite the LHS in the following way 
        \begin{align*}
          1^2 + 2^2 + 3^2 + \cdots + k^2 + (k+1)^2 &= \frac{k(k+1)(2k+1)}{6} + (k+1)^2 \\ 
          &= \frac{k(k+1)(2k+1)}{6} + \frac{6(k+1)^2}{6} \\
          &= \frac{k(k+1)(2k+1)}{6} + \frac{6(k^2 + 2k + 1)}{6} \\ 
          &= \frac{(k^2+k)(2k+1)}{6} + \frac{6k^2 + 12k + 6}{6} \\ 
          &= \frac{2k^3 + 2k^2 + k^2 + k}{6} + \frac{6k^2 + 12k + 6}{6} \\ 
          &= \frac{2k^3 + 9k^2 + 13k + 6}{6}
        \end{align*}
        The RHS can be expanded to give
        \begin{align*}
          \frac{(k+1)(k+2)(2k+3)}{6} &= \frac{(k^2 + 3k + 2)(2k+3)}{6} \\ 
          &= \frac{2k^3 + 3k^2 + 6k^2 + 9k + 4k + 6}{6} \\ 
          &= \frac{2k^3 + 9k^2 + 13k + 6}{6}
        \end{align*}
        Now our LHS and RHS equal the same quantity and must be equivalent, thus our statement holds true through proof by induction.
      \end{proof}
    \end{solution}

    \newpage

    \question Prove the following using proof by induction
    \begin{center}
      For all integers $n \geq 1$, $\displaystyle\, 1 + 3 + 5 + \cdots + (2n-1) = n^2$
    \end{center}
    \begin{solution}[\stretch{1}]
      \begin{proof}
        Let $P(n)$ be the formula 
        \[
        1 + 3 + 5 + \cdots + (2n-1) = n^2
        \]
        \textbf{Show that $P(1)$ is true:} To show that $P(1)$ is true, we will need to show that the LHS and RHS are equivalent. Well, the LHS is $2(1)-1=1$ and the RHS is $1^2=1$ and so both sides equal the same quantity and thus are equivalent, proving the truthfullness of $P(1)$ 
        \newline\textbf{Show that if $P(k)$ is true, so is $P(k+1)$:} We will assume that $P(k)$ is true or that 
        \[
        1 + 3 + 5 + \cdots + (2k-1) = k^2
        \]
        and we must show that $P(k+1)$ is true, or that 
        \[
        1 + 3 + 5 + \cdots + \left(2\left(k+1\right)-1\right) = \left(k+1\right)^2
        \]
        We can begin with the LHS as follows, 
        \begin{align*}
          1 + 3 + 5 + \cdots + (2k + 1) &= 1 + 3 + 5 + \cdots + (2k - 1) + (2k + 1) \\ 
          &= k^2 + (2k + 1) \\ 
          &= (k+1)(k+1) \\ 
          &= \left(k+1\right)^2 
        \end{align*}
        And so, our LHS and RHS equal the same quantity and thus are equivalent, proving our statement through proof by induction.
      \end{proof}
    \end{solution}

    \question Prove the following using proof by induction 
    \begin{center}
      For all integers $n \geq 1$, $\displaystyle\, 2 + 4 + 6 + \cdots + 2n = n^2 + n$
    \end{center}
    \begin{solution}[\stretch{1}]
      \begin{proof}
        Let $P(n)$ be the formula 
        \[
          2 + 4 + 6 + \cdots + 2n = n^2 + n
        \]
        \textbf{Show that $P(1)$ is true:} To show that $P(1)$ is true we will show that the LHS and the RHS are equivalent. The LHS is $2(1)=2$ and the RHS is $1^2 + 1 = 2$ and so, since the LHS and RHS are equivalent, $P(1)$ holds true.
        \newline\textbf{Show that if $P(k)$ is true, so is $P(k+1)$:} We will assume that $P(k)$ is true or that 
        \[
          2 + 4 + 6 + \cdots + 2k = k^2 + k 
        \] 
        and we will show that $P(k+1)$ is true or that 
        \[
          2 + 4 + 6 + \cdots + 2(k+1) = (k+1)^2 + k = k^2 + 2k + 1 + k = k^2 + 3k + 1
        \]
        We can begin with the LHS to show 
        \begin{align*}
          2 + 4 + 6 + \cdots 2(k+1) &= 2 + 4 + 6 + \cdots + 2k + 2(k+1) \\ 
          &= (k^2 + k) + 2(k+1) \\ 
          &= k^2 + k + 2k + 1 \\ 
          &= k^2 + 3k + 1
        \end{align*}
        Which is the RHS of $P(k+1)$ and so our LHS and RHS are equivalent, proving the validity of our statement through proof by induction.
      \end{proof}
    \end{solution}

    \newpage 

    \question \textit{(Challenge)} Prove the following using proof by induction
    \begin{center}
      For all integers $n \geq 1$, $\displaystyle\,\sum\limits_{i=1}^{n} i\left(i\textbf{!}\right) = (n+1)\textbf{!} - 1$
    \end{center}
    \begin{solution}[\stretch{1}]
      \begin{proof}
        Let $P(n)$ be the formula 
        \[
          \sum\limits_{i=1}^{n} i\left(i\textbf{!}\right) = \left(n+1\right)\textbf{!} - 1
        \]
        \textbf{Show that $P(1)$ is true:} To show that $P(1)$ is true we will show that the LHS and RHS are equivalent. 
        \newline
        \begin{minipage}[t]{0.45\linewidth}
          \begin{center}
            \underline{LHS}
          \end{center}
          \[
            \sum\limits_{i=1}^{1} i\left(i\textbf{!}\right) = 1(1\textbf{!}) = 1(1) = 1
          \]
        \end{minipage}
        \hfill
        \begin{minipage}[t]{0.45\linewidth}
          \begin{center}
            \underline{RHS}
          \end{center}
          \[
            \left(1+1\right)\textbf{!} - 1 = 2\textbf{!} - 1 = 2 - 1 = 1
          \]
        \end{minipage}
        \newline And so, our LHS and RHS are equivalent, thus proving $P(1)$ 
        \newline\textbf{Show that if $P(k)$ is true, so is $P(k+1)$:} First, we will assume that $P(k)$ is true or that 
        \[
        \sum\limits_{i=1}^{k} i\left(i\textbf{!}\right) = \left(k+1\right)\textbf{!} - 1
        \]
        and we will need to show that $P(k+1)$ is true or that 
        \[
        \sum\limits_{i=1}^{k+1} i\left(i\textbf{!}\right) = \left[\left(k+1\right)+1\right]\textbf{!}-1 = \left(k+2\right)\textbf{!} - 1
        \]
        We can begin with the LHS as follows, 
        \begin{align*}
          \sum\limits_{i=1}^{k+1} i\left(i\textbf{!}\right) &= \left(k+1\right)\left[\left(k+1\right)\textbf{!}\right] + \sum\limits_{i=1}^{k} i\left(i\textbf{!}\right) \\ 
          &= \left(k+1\right)\left(k+1\right)\textbf{!} + \left[\left(k+1\right)\textbf{!} - 1\right] \\ 
          &= \left[\left(k+1\right)\left(k+1\right)\textbf{!} + \left(k+1\right)\textbf{!}\right] - 1 \tag{by the associative property of addition} \\ 
          &= \left(k+1\right)\textbf{!}\left[\left(k+1\right) + 1\right] - 1 \tag{by factoring}\\ 
          &= \left(k+1\right)\textbf{!}\left(k+2\right) - 1 \\ 
          &= \left(k+2\right)\textbf{!} - 1 \tag{by properties of factorials}
        \end{align*}
        And so, our LHS is equivalent to the same quantity as the RHS and is thus equal to the RHS, proving our statement through induction.
      \end{proof}
    \end{solution}
    \newpage 
    \question \textit{(\textbf{Mega} Challenge)} Prove the following using proof by induction
    \begin{center}
      For all integers $n \geq 1$, $3$ is a factor of $4^n - 1$
    \end{center}
    \begin{solution}[\stretch{1}]
      \begin{proof}
        Let $P(n)$ be the statement that $3$ is a factor of $4^n - 1$ 
        \newline\textbf{Show that $P(1)$ is true:} To show that $P(1)$ is true we will show that $3$ is a factor of $4^1 - 1$. $4^1 - 1 = 3$ and $3/3 = 1$ which is an integer, thus $3$ is a factor of $4^1-1$ and $P(1)$ is valid. 
        \newline\textbf{Show that if $P(k)$ is true, so is $P(k+1)$:} We will assume that $P(k)$ is true or that 
        \[
        \frac{4^k-1}{3} \in \mathbb{Z}
        \]
        and we will need to show that 
        \[
        \frac{4^{k+1}-1}{3} \in \mathbb{Z}
        \]
        We can do this by manipulating the LHS
        \begin{align*}
          4^{k+1} - 1 &= 4^{k+1} - 4^k + 4^k - 1 \\ 
          &= 4^k * 4 - 4^k + 4^k - 1 \\ 
          &= 4^k(4-1) + (4^k - 1) \\ 
          &= 4^k(3) + (4^k - 1)
        \end{align*}
        Since we already know that $4^k-1$ has a factor of $3$ we can conclude that $4^{k+1}$ must also have a factor of $3$ since $\displaystyle\, \frac{4^k(3)}{3} = 4^k \in \mathbb{Z}$ and so adding another factor of $3$ will still give a factor of $3$. Thus, our statement holds true through proof by induction.
      \end{proof}
    \end{solution}
  \end{questions}
\end{document}
