\documentclass[addpoints]{exam}
\usepackage{amsmath}
\usepackage{amsfonts}
\usepackage{amssymb}
\usepackage[most]{tcolorbox}
\usepackage{tikz}
\usepackage{pgfplots}
\usepackage{mdframed}
\usepackage{hyperref}
\usepackage{amsthm}
\usepackage{xcolor}
\usepackage{cancel}

\usetikzlibrary{decorations.markings}

\marksnotpoints
\pointsinrightmargin
\bracketedpoints
%\printanswers

\hypersetup{
  colorlinks=true,
  linkcolor=blue,
  filecolor=magenta,
  urlcolor=blue,
  pdfpagemode=FullScreen,
}

\renewcommand\qedsymbol{$\blacksquare$}
\newtheorem{theorem}{Theorem}

\urlstyle{same}

\pagestyle{headandfoot}
\firstpageheadrule
\runningheadrule
\firstpageheader{Pre Calc Prep}{Advanced Limits}{Shah}
\runningheader{Adv. Limits Notes}{}{Shah}
\firstpagefooter{}{}{}
\runningfooter{ }{\thepage}{ }

\begin{document}
\begin{tcolorbox}[breakable, title=PROPERTIES OF LIMITS, colframe=black, sharp corners, colback=white, colbacktitle=white, coltitle=black]
  Suppose that $\displaystyle\, \lim\limits_{x\to\,c} f(x) = L$ and that $\displaystyle\, \lim\limits_{x\to\,c} g(x) = K$, 
  \begin{enumerate}
    \item $\displaystyle\,n * \lim\limits_{x\to\,c} f(x) = \lim\limits_{x\to\,c} n*f(x) = n*L$
    \item $\displaystyle\,\lim\limits_{x\to\,c} \left[f(x)\right] \pm \lim\limits_{x\to\,c} \left[g(x)\right] = \lim\limits_{x\to\,c} \left[f(x) \pm g(x)\right] = L \pm K$
    \item $\displaystyle\, \lim\limits_{x\to\,c} \left[f(x)*g(x)\right] = \lim\limits_{x\to\,c} \left[f(x)\right] * \lim\limits_{x\to\,c} \left[g(x)\right] = L * K$
    \item $\displaystyle \lim\limits_{x\to\,c} \left[\frac{f(x)}{g(x)}\right] = \frac{\lim\limits_{x\to\,c} \left[f(x)\right]}{\lim\limits_{x\to\,c} \left[g(x)\right]} = \frac{L}{K}, \hspace{0.1in} \mbox{provided } \lim\limits_{x\to\,c} \left[g(x)\right] \ne 0$
    \item $\displaystyle \lim\limits_{x\to\,c} \left[f(x)\right]^{n} = \left[\lim\limits_{x\to\,c} f(x)\right]^n = L^n, \hspace{0.1in} \mbox{where $n$ is any real number}$
    \item $\displaystyle\, \lim\limits_{x\to\,c} a = a, \hspace{0.2in} \mbox{$a$ is any real number}$
    \item $\displaystyle\, \lim\limits_{x\to\,c} x = c$
    \item $\displaystyle\, \lim\limits_{x\to\,c} x^n = c^n$
  \end{enumerate}
\end{tcolorbox}

\begin{questions}
  \question Evaluate the following limit using \textit{only} the properties above
  \[\lim\limits_{x\to\,-2} \left(3x^2+5x-9\right)\]
  \begin{solution}[1.75in]
    \begin{align*}
      \lim\limits_{x\to\,-2} \left(3x^2 + 5x - 9\right) &= \lim\limits_{x\to\,-2} \left(3x^2\right) + \lim\limits_{x\to\,-2} \left(5x\right) + \lim\limits_{x\to\,-2} \left(-9\right) \tag{by property 2} \\ 
      &= 3\lim\limits_{x\to\,-2} \left(x^2\right) + 5\lim\limits_{x\to\,-2} \left(x\right) - \lim\limits_{x\to\,-2} \left(9\right) \tag{by property 1} \\ 
      &= 3\left(-2\right)^2 + 5\left(-2\right) - 9 \tag{by properties 6-8} \\ 
      &= 3(4) - 10 - 9 = \boxed{-7}
    \end{align*}
  \end{solution}
\end{questions}
\noindent\makebox[\linewidth]{\hrulefill}
Now note that if defined $\displaystyle\,p(x) = 3x^2+5x-9$ and evaluated $\lim\limits_{x\to\,-2} p(x)$ a quick direct substitution would have yielded that our limit equals $p(-2)=-7$
\end{document}
